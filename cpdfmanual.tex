%Document %URL - does not work for diagonal
%Document -font-ttf
%Document -draw and friends
%Document -png -jpeg
%Document -list-annotations[-json] now obey page range
%Document round-tripping of annotations, supersede -copy-annotations.
%Document -utf for JSON and mark -clean-strings as deprecated since can fail to round-trip binary strings which begin with a BOM?
%Document -composition[-json]
%Document discourage GhostScript usage, since it can strip data (-gs-malformed, embed missing fonts)
%Document [ ] pagespecs
%Document extensions to -info
\documentclass{book}
% Edit here to produce cpdfmanual.pdf, cpdflibmanual.pdf, pycpdfmanual.pdf,
% dotnetcpdflibmanual.pdf, jcpdflibmanual.pdf jscpdflibmanual.pdf etc.
\usepackage{comment}
\excludecomment{cpdflib}
\excludecomment{pycpdflib}
\excludecomment{dotnetcpdflib}
\excludecomment{jcpdflib}
\excludecomment{jscpdflib}
\usepackage{alltt}
\usepackage{palatino}
\usepackage{listings}
\usepackage{microtype}
\usepackage{graphics}
\usepackage{upquote}
\usepackage[plainpages=false,pdfpagelabels,pdfborder=0 0 0,draft=false,hidelinks,bookmarksnumbered]{hyperref}
\usepackage{framed}
\newcommand{\smallgap}{\bigskip}
\newcommand{\cpdf}{\texttt{cpdf}}
\addtolength{\textwidth}{20mm}
\usepackage{makeidx}\makeindex
\usepackage[left=3cm, right=1.5cm, top=2cm, bottom=1.8cm, paperwidth=7.5in, paperheight=9.25in]{geometry}
\usepackage{fancyhdr}
\fancyhf{}
\pagestyle{fancy}
\fancyhead[lo]{\slshape\nouppercase{\leftmark}\hfill\thepage}
\fancyhead[re]{\thepage\hfill\slshape\nouppercase{\leftmark}}
\fancyfoot{}
\renewcommand{\headrulewidth}{0pt}
\renewcommand{\footrulewidth}{0pt}
\begin{document}

\frontmatter
\pagestyle{empty}


\begin{flushright}


{\sffamily \bfseries \Huge Coherent PDF

\vspace{2mm}
\begin{cpdflib}
C API and 
\end{cpdflib}
\begin{pycpdflib}
Python API and
\end{pycpdflib}
\begin{dotnetcpdflib}
.NET API and
\end{dotnetcpdflib}
\begin{jcpdflib}
Java API and
\end{jcpdflib}
\begin{jscpdflib}
JavaScript API and
\end{jscpdflib}
Command Line Tools}

\vspace{12mm}

{\Huge User Manual}\\
Version 2.5 (January 2022)

\vspace{25mm}

\vfill
\ifdefined\HCode
  %htlatex code here
\else
   %pdflatex code here
\includegraphics{logo.pdf}
\fi


\vspace{2mm}
{\sffamily \bfseries \LARGE Coherent Graphics Ltd}

\end{flushright}

\clearpage

\pagestyle{empty}
\noindent For bug reports, feature requests and comments, email\\ \texttt{contact@coherentgraphics.co.uk}

\vspace*{\fill}
\noindent\copyright 2022 Coherent Graphics Limited. All rights reserved. ISBN 978-0957671140

\smallgap 
\noindent Adobe, Acrobat, and Adobe PDF are
registered trademarks of Adobe Systems Incorporated. Windows, Powerpoint and
Excel are registered trademarks of Microsoft Corporation.

\cleardoublepage

\pagestyle{plain}
\chapter*{Quickstart Examples}

Just a few of the facilities provided by the Coherent PDF Command Line Tools. See each chapter for more commands and full details.

\section*{\hyperref[chap:1]{Chapter 1: Basic Usage}}

\begin{framed}\noindent\texttt{cpdf in.pdf 1-3,6 -o out.pdf}\end{framed}

\noindent Read \texttt{in.pdf}, select pages 1, 2, 3 and 6, and write those pages to \texttt{out.pdf}.


\begin{framed}\noindent\texttt{cpdf in.pdf even -o out.pdf}\end{framed}
                  
\noindent Select the even pages (2, 4, 6...) from \texttt{in.pdf} and write those pages to \texttt{out.pdf}.

\begin{framed}\noindent\texttt{cpdf -merge in.pdf in2.pdf AND -add-text "Copyright 2021"\\\phantom{\ \ \ \ } -o out.pdf}\end{framed}

\noindent Using \texttt{AND} to perform several operations in order, here merging two files together and adding a copyright stamp to every page.

\begin{framed}\noindent\texttt{cpdf -args control.txt}\end{framed}

\noindent Read \texttt{control.txt} and use its contents as the command line arguments for \texttt{cpdf}.

\section*{\hyperref[chap:2]{Chapter 2: Merging and Splitting}}

\begin{framed}\noindent\texttt{cpdf -merge in.pdf in2.pdf -o out.pdf}\end{framed}

\noindent Merge \texttt{in.pdf} and \texttt{in2.pdf} into one document, writing to \texttt{out.pdf}.

\begin{framed}\noindent\texttt{cpdf -split in.pdf -o Chunk\%\%\%.pdf -chunk 10}\end{framed}

\noindent Split \texttt{in.pdf} into ten-page chunks, writing them to \texttt{Chunk001.pdf}, \texttt{Chunk002.pdf} etc.

\begin{framed}\noindent\texttt{cpdf -split-bookmarks 0 in.pdf -o @B.pdf}\end{framed}

\noindent Split \texttt{in.pdf} on bookmark boundaries, writing each to a file whose name is the bookmark label.

\section*{\hyperref[chap:3]{Chapter 3: Pages}}

\begin{framed}\noindent\texttt{cpdf -scale-page "2 2" in.pdf -o out.pdf}\end{framed}

\noindent Scale both the dimensions and contents of \texttt{in.pdf} by a factor of two in x and y directions.

\begin{framed}\noindent\texttt{cpdf -scale-to-fit usletterportrait in.pdf -o out.pdf}\end{framed}

\noindent Scale the pages in \texttt{in.pdf} to fit the US Letter page size, writing to \texttt{out.pdf}

\begin{framed}\noindent\texttt{cpdf -shift "26pt 18mm" in.pdf -o out.pdf}\end{framed}

\noindent Shift the contents of the page by 26 pts in the x direction, and 18 millimetres in the y direction, writing to \texttt{out.pdf}

\begin{framed}\noindent\texttt{cpdf -rotate-contents 90 in.pdf -o out.pdf}\end{framed}

\noindent Rotate the contents of the pages in \texttt{in.pdf} by ninety degrees and write to \texttt{out.pdf}.

\begin{framed}\noindent\texttt{cpdf -cropbox "0 0 600pt 400pt" in.pdf -o out.pdf}\end{framed}

\noindent Crop the pages in \texttt{in.pdf} to a 600 pts by 400 pts rectangle.


\section*{\hyperref[chap:4]{Chapter 4: Encryption and Decryption}}

\begin{framed}\noindent\texttt{cpdf -encrypt 128bit fred joe in.pdf -o out.pdf}\end{framed}

\noindent Encrypt \texttt{in.pdf} using 128bit PDF encryption using the owner password \texttt{fred} and the user password \texttt{joe} and writing the encrypted file to \texttt{out.pdf}

\begin{framed}\noindent\texttt{cpdf -decrypt in.pdf owner=fred -o out.pdf}\end{framed}

\noindent Decrypt \texttt{in.pdf} using the owner password, writing to \texttt{out.pdf}.


\section*{\hyperref[chap:5]{Chapter 5: Compression}}

\begin{framed}\noindent\texttt{cpdf -compress in.pdf -o out.pdf}\end{framed}

\noindent Compress the data streams in \texttt{in.pdf}, writing the result to \texttt{out.pdf}.

\begin{framed}\noindent\texttt{cpdf -decompress in.pdf -o out.pdf}\end{framed}

\noindent Decompress the data streams in \texttt{in.pdf}, writing to \texttt{out.pdf}.

\begin{framed}\noindent\texttt{cpdf -squeeze in.pdf -o out.pdf}\end{framed}

\noindent Squeeze \texttt{in.pdf}, writing to \texttt{out.pdf}. Squeezing rearranges the structure of the PDF file to save space.


\section*{\hyperref[chap:6]{Chapter 6: Bookmarks}}

\begin{framed}\noindent\texttt{cpdf -list-bookmarks in.pdf}\end{framed}

\noindent List the bookmarks in \texttt{in.pdf}. 

\begin{framed}\noindent\texttt{cpdf -add-bookmarks bookmarks.txt in.pdf -o out.pdf}\end{framed}

\noindent Add bookmarks in the same form from a prepared file \texttt{bookmarks.txt} to \texttt{in.pdf}, writing to \texttt{out.pdf}.


\section*{\hyperref[chap:7]{Chapter 7: Presentations}}

\begin{framed}\noindent\texttt{cpdf -presentation in.pdf 2-end -trans Split -duration 10\\\phantom{\ \ \ \ } -o out.pdf}\end{framed}

\noindent Use the Split style to build a presentation from the PDF \texttt{in.pdf}, each slide staying 10 seconds on screen unless manually advanced. The first page, being a title does not move on automatically, and has no transition effect.

\section*{\hyperref[chap:8]{Chapter 8: Logos, Watermarks and Stamps}}

\begin{framed}\noindent\texttt{cpdf -stamp-on watermark.pdf in.pdf -o out.pdf}\end{framed}

\noindent Stamp the file \texttt{watermark.pdf} on to each page of \texttt{in.pdf}, writing the result to \texttt{out.pdf}.

\begin{framed}\noindent\texttt{cpdf -topleft 10 -font Courier\\ \phantom{\ \ \ \ } -add-text "Page \%Page\textbackslash nDate \%d-\%m-\%Y" in.pdf -o out.pdf}\end{framed}

\noindent Add a page number and date to all the pages in \texttt{in.pdf} using the Courier font, writing to \texttt{out.pdf}.

\section*{\hyperref[chap:9]{Chapter 9: Multipage Facilities}}

\begin{framed}\noindent\texttt{cpdf -twoup-stack in.pdf -o out.pdf}\end{framed}

\noindent Two up impose the file \texttt{in.pdf}, writing to \texttt{out.pdf}.

\begin{framed}\noindent\texttt{cpdf -pad-after in.pdf 1,3,4 -o out.pdf}\end{framed}

\noindent Add extra blank pages after pages one, three and four of a document.

\section*{\hyperref[chap:10]{Chapter 10: Annotations}}

\begin{framed}\noindent\texttt{cpdf -list-annotations in.pdf}\end{framed}

\noindent List the annotations in a file \texttt{in.pdf} to standard output.

\begin{framed}\noindent\texttt{cpdf -copy-annotations from.pdf in.pdf -o out.pdf}\end{framed}

\noindent Copy the annotations from \texttt{from.pdf} to \texttt{in.pdf}, writing to \texttt{out.pdf}.

\section*{\hyperref[chap:11]{Chapter 11: Document Information and Metadata}}

\begin{framed}\noindent\texttt{cpdf -set-title "The New Title" in.pdf -o out.pdf}\end{framed}

\noindent Set the document title of \texttt{in.pdf}. writing to \texttt{out.pdf}.

\begin{framed}\noindent\texttt{cpdf -hide-toolbar true in.pdf -o out.pdf}\end{framed}

\noindent Set the document \texttt{in.pdf} to open with the PDF Viewer's toolbar hidden, writing to \texttt{out.pdf}.

\begin{framed}\noindent\texttt{cpdf -set-metadata metadata.xml in.pdf -o out.pdf}\end{framed}

\noindent Set the metadata in a PDF \texttt{in.pdf} to the contents of the file \texttt{metadata.xml}, and write the output to \texttt{out.pdf}.

\begin{framed}\noindent\texttt{cpdf -set-page-layout TwoColumnRight in.pdf -o out.pdf}\end{framed}

\noindent Set the document \texttt{in.pdf} to open in PDF Viewer showing two columns of pages, starting on the right, putting the result in \texttt{out.pdf}.

\begin{framed}\noindent\texttt{cpdf -set-page-mode FullScreen in.pdf -o out.pdf}\end{framed}

\noindent Set the document \texttt{in.pdf} to open in PDF Viewer in full screen mode, putting the result in \texttt{out.pdf}.

\section*{\hyperref[chap:12]{Chapter 12: File Attachments}}

\begin{framed}\noindent\texttt{cpdf -attach-file sheet.xls in.pdf -o out.pdf}\end{framed}

\noindent Attach the file \texttt{sheet.xls} to \texttt{in.pdf}, writing to \texttt{out.pdf}.

\begin{framed}\noindent\texttt{cpdf -remove-files in.pdf -o out.pdf}\end{framed}

\noindent Remove any attachments from \texttt{in.pdf}, writing to \texttt{out.pdf}.

\section*{\hyperref[chap:13]{Chapter 13: Working with Images}}

\begin{framed}\noindent\texttt{cpdf -image-resolution 600 in.pdf}\end{framed}

\noindent Identify and list any image used at less than 600dpi.

\begin{framed}\noindent\texttt{cpdf -extract-images in.pdf -im /usr/bin/magick -o output/\%\%\%}\end{framed}

\noindent Extract images from \texttt{in.pdf} to directory \texttt{output} (with the help of imagemagick).


\section*{\hyperref[chap:14]{Chapter 14: Fonts}}

\begin{framed}\noindent\texttt{cpdf -list-fonts in.pdf}\end{framed}

\noindent List the fonts in use, and what pages they are used on.

\begin{framed}\noindent\texttt{cpdf -gs /usr/bin/gs -embed-missing-fonts in.pdf -o out.pdf}\end{framed}

\noindent Embed missing fonts (with the help of Ghostscript).

\section*{\hyperref[chap:15]{Chapter 15: PDF and JSON}}

\begin{framed}\noindent\texttt{cpdf in.pdf -output-json -output-json-parse-content-streams\\\phantom{\ \ \ \ } -o out.json}\end{framed}

\noindent Write the PDF in JSON format to the given file, parsing its content streams into individual JSON objects too.

\begin{framed}\noindent\texttt{cpdf -j in.json -o out.pdf}\end{framed}

\noindent Load a PDF in JSON format, writing to an output PDF.

\section*{\hyperref[chap:16]{Chapter 16: Optional Content Groups}}

\begin{framed}\noindent\texttt{cpdf -ocg-list in.pdf}\end{framed}

\noindent List the optional content groups by name.

\begin{framed}\noindent\texttt{cpdf -ocg-coalesce-on-name in.pdf -o out.pdf}\end{framed}

\noindent Coalesce optional content groups after merging or stamping two files with OCGs with like names.

\section*{\hyperref[chap:17]{Chapter 17: Creating New PDFs}}

\begin{framed}
 \small\verb?cpdf -create-pdf -create-pdf-pages 20?\\
 \noindent\small\verb?        -create-pdf-papersize usletterportrait -o out.pdf?
\end{framed}

\noindent Create a US Letter PDF of twenty pages.

\begin{framed}
 \small\verb?cpdf -typeset file.txt -create-pdf-papersize a3portrait?\\
 \noindent\small\verb?        -font Courier -font-size 10 -o out.pdf?
\end{framed}

\noindent Typeset a text file as PDF on A3 paper with Courier 10 point font.


\section*{\hyperref[chap:misc]{Chapter 18: Miscellaneous}}

\begin{framed}\noindent\texttt{cpdf -blacktext in.pdf -o out.pdf}\end{framed}

\noindent Blacken all the text in \texttt{in.pdf}, writing to \texttt{out.pdf}.

\begin{framed}\noindent\texttt{cpdf -thinlines 2pt in.pdf -o out.pdf}\end{framed}

\noindent Make sure all lines in \texttt{in.pdf} are at least 2 pts wide, writing to \texttt{out.pdf}.
% Letter
\pagestyle{empty}
\cleardoublepage

\begin{cpdflib}
\chapter*{Example Program in C}

This program loads a file \texttt{hello.pdf} from disk and writes out a document with the original included three times. Note the use of \texttt{cpdf\_startup}, \texttt{cpdf\_lastError} and \texttt{cpdf\_clearError}.

\begin{small}
\begin{verbatim}
#include <stdbool.h>
#include "cpdflibwrapper.h"

int main (int argc, char ** argv)
{
  /* Initialise cpdf */
  cpdf_startup(argv);

  /* We will take the input hello.pdf and repeat it three times */
  int mergepdf = cpdf_fromFile("hello.pdf", "");

  /* Check the error state */
  if (cpdf_lastError) return 1;

  /* Clear the error state */
  cpdf_clearError();

  /* The array of PDFs to merge */
  int pdfs[] = {mergepdf, mergepdf, mergepdf};

  /* Merge them */
  int merged = cpdf_mergeSimple(pdfs, 3);
  
  if (cpdf_lastError) return 1;

  cpdf_clearError();

  /* Write output */
  cpdf_toFile(merged, "merged.pdf", false, false);

  if (cpdf_lastError) return 1;

  return 0;
}
\end{verbatim}
\end{small}
\end{cpdflib}

\begin{pycpdflib}
\chapter*{Example Program in Python}

This program loads a file \texttt{hello.pdf} from disk and writes out a
document with the original included three times.

\begin{small}
\begin{verbatim}
#Merge example
import pycpdflib

#DLL loading depends on your own platform. These are the author's settings.
if sys.platform.startswith('darwin'):
    pycpdflib.loadDLL("/Users/john/repos/python-libcpdf/libpycpdf.so")
elif sys.platform.startswith('linux'):
    pycpdflib.loadDLL("../libpycpdf.so")
elif sys.platform.startswith('win32') or sys.platform.startswith('cygwin'):
    os.add_dll_directory("C:\\\\OCaml64/home/JohnWhitington/python-libcpdf/")
    pycpdflib.loadDLL("libpycpdf.dll")

#We will take the input hello.pdf and repeat it three times
mergepdf = pycpdf.fromFile('hello.pdf', '')

#The list of PDFs to merge
pdfs = [mergepdf, mergepdf, mergepdf]

#Merge them
merged = pycpdflib.mergeSimple(pdfs)

#Write output
pycpdflib.toFile(merged, 'merged.pdf', False, False)
\end{verbatim}
\end{small}
\end{pycpdflib}

\begin{dotnetcpdflib}
\chapter*{Example Program in C\#}

This program loads a file \texttt{hello.pdf} from disk and writes out a
document with the original included three times.

\begin{small}
\begin{verbatim}
//Merge example
using System;
using System.Collections.Generic;
using CoherentGraphics;

// Initialise cpdf
Cpdf.startup();

// We will take the input hello.pdf and repeat it three times
using (Cpdf.Pdf mergepdf = Cpdf.fromFile("hello.pdf", ""))
{
  // The list of PDFs to merge
  List<Cpdf.Pdf> pdfs = new List<Cpdf.Pdf> {mergepdf, mergepdf, mergepdf};

  // Merge them
  Cpdf.Pdf merged = Cpdf.mergeSimple(pdfs);

  // Write output
  Cpdf.toFile(merged, "merged.pdf", false, false);
  
  // Dispose of merged PDF
  merged.Dispose();
}
\end{verbatim}
\end{small}

\noindent Note the use of \texttt{using} and \texttt{Dispose()} to ensure the PDFs are thrown away when no longer required.

\chapter*{Example Program in VB.NET}

This program loads a file \texttt{hello.pdf} from disk and writes out a
document with the original included three times.

\begin{small}
\begin{verbatim}
' Merge example
imports System
imports System.Collections.Generic
imports CoherentGraphics

' Initialise cpdf
Cpdf.startup()

' We will take the input hello.pdf and repeat it three times
Using mergepdf As Cpdf.Pdf = Cpdf.fromFile("hello.pdf", "")
  ' The list of PDFs to merge
  Dim pdfs As List(Of Cpdf.Pdf) =
    new List(Of Cpdf.Pdf)({mergepdf, mergepdf, mergepdf})

  ' Merge them
  Dim merged As Cpdf.Pdf = Cpdf.mergeSimple(pdfs)

  ' Write output
  Cpdf.toFile(merged, "merged.pdf", false, false)

  ' Dispose of merged PDF
  merged.Dispose()
End Using
\end{verbatim}
\end{small}

\noindent Note the use of \texttt{Using} and \texttt{Dispose()} to ensure the PDFs are thrown away when no longer required.

\end{dotnetcpdflib}

\begin{jcpdflib}
\chapter*{Example Program in Java}

This program loads a file \texttt{hello.pdf} from disk and writes out a
document with the original included three times.

\begin{small}
\begin{verbatim}
//Merge example
import com.coherentpdf.Jcpdf

public static void main(String[] args)
{
   // Initialise cpdf
   Jcpdf jcpdf = new Jcpdf();
   try
   {
     jcpdf.startup();
   }
   catch (Jcpdf.CpdfError e)
   {
     System.out.println("Error during cpdf startup");
   }
   // We will take the input hello.pdf and repeat it three times
   try (Jcpdf.Pdf mergepdf = jcpdf.fromFile("hello.pdf", ""))
   {
     // The array of PDFs to merge
     Jcpdf.Pdf[] pdfs = {mergepdf, mergepdf, mergepdf};
     // Merge them
     Jcpdf.Pdf merged = jcpdf.mergeSimple(pdfs);
     // Write output
     jcpdf.toFile(merged, "merged.pdf", false, false);
     // Dispose of merged PDF
     merged.close();
   }
   catch (Jcpdf.CpdfError e)
   {
     System.out.println("Error during cpdf operation");
   }
}
\end{verbatim}
\end{small}

\noindent Note the use of \texttt{try} and \texttt{close()} to ensure the PDFs are thrown away when no longer required.
\end{jcpdflib}

\begin{jscpdflib}
\chapter*{Example Program in JavaScript}

This program loads a file \texttt{hello.pdf} from disk and writes out a
document with the original included three times.

\begin{small}
\begin{verbatim}
//Merge example

//Load coherentpdf.js
const coherentpdf = require('./coherentpdf.js');

//Load the file hello.pdf from the current directory
var pdf = coherentpdf.fromFile('hello.pdf', '');

//Merge three copies of it
var merged = coherentpdf.mergeSimple([pdf, pdf, pdf]);

//Write to merged.pdf
coherentpdf.toFile(merged, 'merged.pdf', false, false);

//Clean up the two PDFs
coherentpdf.deletePdf(pdf);
coherentpdf.deletePdf(merged);
\end{verbatim}
\end{small}

\noindent To be run in node. A browser example is included in the distribution of coherentpdf.js.
\end{jscpdflib}

\pagestyle{plain}
\tableofcontents\clearpage\pagestyle{empty}

\cleardoublepage
\pagestyle{plain}
\chapter*{Typographical Conventions}
Command lines to be typed are shown in \texttt{typewriter\hspace{-1mm} font} in a box.
For example:
\begin{framed}
\small\verb!cpdf in.pdf -o out.pdf!
\end{framed}
\noindent When describing the general form of a command, rather than a particular
example, square brackets \verb|[]| are used to enclose optional parts, and
angled braces \verb!<>! to enclose general descriptions which may be
substituted for particular instances. For example,
\begin{framed}
\small\verb!cpdf <operation> in.pdf [<range>] -o out.pdf!
\end{framed}
\noindent describes a command line which requires an operation and, optionally,
a range. An exception is that we use \texttt{in.pdf} and \texttt{out.pdf}
instead of \texttt{<input file>} and \texttt{<output file>} to reduce
verbosity. Under Microsoft Windows, type \texttt{cpdf.exe} instead of \texttt{cpdf}.
\clearpage\pagestyle{empty}\cleardoublepage
\mainmatter
%\chapterstyle{hangnum}
%\pagestyle{ruled}
\pagestyle{fancy}



\chapter{Basic Usage}\label{chap:1}


\label{basicusage}
  \begin{framed}
  \small
  \noindent\begin{verbatim}
  -help                   --help                   -version 
  -o                      -i                       -idir <directory>
  -recrypt                -decrypt-force           -stdout
  -stdin                  -stdin-user <password>   -stdin-owner <password>  
  -producer <text>        -creator <text>          -change-id
  -l                      -cpdflin <filename>      -keep-l
  -no-preserve-objstm     -create-objstm           -control <filename>
  -args <filename>        -utf8                    -stripped
  -raw                    -no-embed-font           -gs
  -gs-malformed           -gs-malformed-force      -gs-quiet
  -error-on-malformed\end{verbatim}\end{framed}

  The Coherent PDF tools provide a wide range of facilities for modifying PDF
files created by other means. There is a single command-line program
\cpdf\ (\texttt{cpdf.exe} under Microsoft Windows). The rest of this manual describes the options that may be given
to this program.

\section{Documentation}

The operation \texttt{-help / --help} prints each operation and option together with a short description. The operation \texttt{-version} prints the cpdf version string.

  \index{input files} \index{output files}
  \section{Input and Output Files}
  The typical pattern for usage is
  \begin{framed}
  \small\verb!cpdf [<operation>] <input file(s)>  -o <output file>!
  \end{framed}
  \noindent and the simplest concrete example, assuming the existence of a file
\texttt{in.pdf} is:
  \begin{framed}
  \small\verb!cpdf in.pdf -o out.pdf!
  \end{framed}
  \noindent which copies \texttt{in.pdf} to \texttt{out.pdf}. The input and
output may be the same file. Of course, we should like to do more interesting
things to the PDF file than that!

  Files on the command line are distinguished from other input by their
containing a period. If an input file does not contain a period, it should be
preceded by \verb!-i!. For example:

  \begin{framed}
  \small\verb!cpdf -i in -o out.pdf!
  \end{framed}
\noindent A whole directory of files may be added (where a command supports multiple files) by using the \verb!-idir! option:
  \begin{framed}
  \small\verb!cpdf -merge -idir myfiles -o out.pdf!
  \end{framed}
  \noindent The files in the directory \verb!myfiles! are considered in alphabetical order. They must all be PDF files. If the names of the files are numeric, leading zeroes will be required for the order to be correct (e.g \verb!001.pdf!, \verb!002.pdf! etc).

To restrict cpdf to files ending in \texttt{.pdf} (in upper or lower or mixed case) add the option \texttt{-idir-only-pdfs} \textit{before} \texttt{-idir}:
  \begin{framed}
  \small\verb!cpdf -merge -idir-only-pdfs -idir myfiles -o out.pdf!
  \end{framed}


  \section{Input Ranges}
  An \index{input range} \index{range} \textit{input range} may be specified
after each input file. This is treated differently by each operation. For
instance
  \begin{framed}
  \small\verb!cpdf in.pdf 2-5 -o out.pdf!
  \end{framed}
  \noindent extracts pages two, three, four and five from \texttt{in.pdf},
writing the result to \texttt{out.pdf}, assuming that \texttt{in.pdf} contains
at least five pages.
\index{page!range}
\index{reversing}
  Here are the rules for building input ranges:
  \begin{itemize}
    \item A dash (\texttt{-}) defines ranges, e.g. \texttt{1-5} or \texttt{6-3}.
    \item A comma (\texttt{,}) allows one to specify several ranges, e.g. \texttt{1-2,4-5}.
    \item The word \texttt{end} represents the last page number.
    \item The words \texttt{odd} and \texttt{even} can be used in place of or at the end of a page range to restrict to just the odd or even pages.
    \item The words \texttt{portrait} and \texttt{landscape} can be used in place of or at the end of a page range to restrict to just those pages which are portrait or landscape. Note that the meaning of ``portrait'' and ``landscape'' does not take account of any viewing rotation in place (use \texttt{-upright} from chapter 3 first, if required). A page with equal width and height is considered neither portrait nor landscape.
    \item The word \texttt{reverse} is the same as \texttt{end-1}.
    \item The word \texttt{all} is the same as \texttt{1-end}.
    \item A range must contain no spaces.
    \item A tilde (\texttt{\~{}}) defines a page number counting from the end of the document rather than the beginning. Page \texttt{\~{}1} is the last page, \texttt{\~{}2} the penultimate page etc.
    \item Prepending \texttt{NOT} to a whole page range inverts it. 
    \item Prepending \verb!<n>!\texttt{DUP} to a whole page range duplicates each page of the range \verb!<n>! times.\index{page!duplicate}
  \end{itemize}
 
  \noindent For example:
  \begin{framed}
  \small\verb!cpdf in.pdf 1,2,7-end -o out.pdf!

  \vspace{2.5mm}
  \noindent Remove pages three, four, five and six from a document.

  \vspace{2.5mm}
  \verb!cpdf in.pdf 1-16odd -o out.pdf!

  \vspace{2.5mm}
  \noindent Extract the odd pages 1,3,...,13,15.
  
  \vspace{2.5mm}
  \verb!cpdf in.pdf landscape -rotate 90 -o out.pdf!

  \vspace{2.5mm}
  \noindent Rotate all landscape pages by ninety degrees.
    
  \vspace{2.5mm}
  \verb!cpdf in.pdf 1,all -o out.pdf!

  \vspace{2.5mm}
  \noindent Duplicate the front page of a document, perhaps as a fax cover sheet.

  \vspace{2.5mm}
  \verb!cpdf in.pdf ~3-~1 -o out.pdf!

  \vspace{2.5mm}
  \noindent Extract the last three pages of a document, in order.

  \vspace{2.5mm}
  \verb!cpdf in.pdf 2DUP1-10 -o out.pdf!

  \vspace{2.5mm}
  \noindent Produce the pages 1,1,2,2,....10,10.

  \end{framed}

\index{decryption}
  \section{Working with Encrypted Documents}
\index{owner password}
\index{user password}
\index{password}
  In order to perform many operations, encrypted input PDF files must be
decrypted. Some require the owner password, some either the user or owner
passwords. Either password is supplied by writing \texttt{user=<password>} or
\texttt{owner=<password>} following each input file requiring it (before or
after any range). The document will \textit{not} be re-encrypted upon writing.  For
example:

\begin{framed}
\small\verb!cpdf in.pdf user=charles -info!\\
\indent\small\verb!cpdf in.pdf owner=fred reverse -o out.pdf!
\end{framed}

\noindent To re-encrypt the file with its existing encryption upon writing, which is required if only the user password was supplied, but allowed in any case, add the \texttt{-recrypt} option:

\begin{framed}
\small\verb!cpdf in.pdf user=charles reverse -recrypt -o out.pdf!
\end{framed}

\noindent The password required (owner or user) depends upon the operation
being performed. Separate facilities are provided to decrypt and encrypt files
(See Section \ref{crypt}).

When appropriate passwords are not available, the option \texttt{-decrypt-force} may be added to the command line to process the file regardless.

For decryption with AES256, passwords may be Unicode. However, the password, should it contain non-ASCII characters, must be normalized by applying the SASLPrep profile (RFC 4013) of the stringprep algorithm (RFC 3454) using the Normalize and BiDi options. It must then be converted to UTF8 and truncated to 127 bytes. Cpdf does not perform this pre-processing - it takes its passwords from the command line without processing.

  \section{Standard Input and Standard Output}
\index{standard input} \index{standard output}
  Thus far, we have assumed that the input PDF will be read from a file on
disk, and the output written similarly. Often it's useful to be able to read
input from \texttt{stdin} (Standard Input) or write output to \texttt{stdout}
(Standard Output) instead. The typical use is to join several programs
together into a \textit{pipe}, passing data from one to the next without the
use of intermediate files. Use \texttt{-stdin} to read from standard input, and
\texttt{-stdout} to write to standard input, either to pipe data between
multiple programs, or multiple invocations of the same program. For example, this sequence of commands (all typed on one line)

  \begin{framed}
  \small\begin{verbatim}   cpdf in.pdf reverse -stdout |
   cpdf -stdin 1-5 -stdout |
   cpdf -stdin reverse -o out.pdf\end{verbatim}
  \end{framed}

\noindent extracts the last five pages of \texttt{in.pdf} in the correct order,
writing them to \texttt{out.pdf}. It does this by reversing the input, taking
the first five pages and then reversing the result.

To supply passwords for a file from \texttt{-stdin}, use \texttt{-stdin-owner <password>} and/or \texttt{-stdin-user <password>}.

Using \texttt{-stdout} on the final command in the pipeline to output the PDF
to screen is not recommended, since PDF files often contain compressed sections
which are not screen-readable.

Several \cpdf\ operations write to standard output by default (for
example, listing fonts). A useful feature of the command line (not specific to
\cpdf) is the ability to redirect this output to a file. This is
achieved with the \texttt{>} operator:

\begin{framed}
  \small\verb!cpdf -info in.pdf > file.txt!

  \vspace{2.5mm}
  \noindent Use the \texttt{-info} operation (See Section \ref{info}), redirecting the
output to \texttt{file.txt}.
\end{framed}

\section{Doing Several Things at Once with AND}
\index{AND}
The keyword \texttt{AND} can be used to string together several commands in
one. The advantage compared with using pipes is that the file need not be
repeatedly parsed and written out, saving time.

To use \texttt{AND}, simply leave off the output specifier (e.g \texttt{-o}) of
one command, and the input specifier (e.g filename) of the next. For instance:

\begin{framed}
  \small\verb!cpdf -merge in.pdf in2.pdf AND -add-text "Label"!
  \noindent\small\verb!        AND -merge in3.pdf -o out.pdf!

  \vspace{2.5mm}
  \noindent Merge \texttt{in.pdf} and \texttt{in2.pdf} together, add text to both pages, append \texttt{in3.pdf} and write to \texttt{out.pdf}.
\end{framed}

\noindent To specify the range for each section, use \texttt{-range}:

\begin{framed}
  \small\verb!cpdf -merge in.pdf in2.pdf AND -range 2-4 -add-text "Label"!
  \noindent\small\verb!        AND -merge in3.pdf -o out.pdf!

\end{framed}

\section{Units}
\index{units}
When measurements are given to \cpdf, they are in points (1 point = 1/72 inch). They may optionally
be followed by some letters to change the measurement. The following are
supported:

\begin{center}
\begin{tabular}{rl}
  \texttt{pt} & Points (72 points per inch). The default. \\
  \texttt{cm} & Centimeters \\
  \texttt{mm} & Millimeters \\
  \texttt{in} & Inches \\
\end{tabular}
\end{center}

\noindent For example, one may write \texttt{14mm} or \texttt{21.6in}. In addition, the following letters stand, in some operations (\texttt{-scale-page}, \texttt{-scale-to-fit}, \texttt{-scale-contents}, \texttt{-shift}, \texttt{-mediabox},\\ \texttt{-crop}) for various page dimensions:

\begin{center}
\begin{tabular}{rl}
  \texttt{PW} & Page width\\
  \texttt{PH} & Page height\\
  \texttt{PMINX} & Page minimum x coordinate\\
  \texttt{PMINY} & Page minimum y coordinate\\
  \texttt{PMAXX} & Page maximum x coordinate\\
  \texttt{PMAXY} & Page maximum y coordinate\\
  \texttt{CW} & Crop box width\\
  \texttt{CH} & Crop box height\\
  \texttt{CMINX} & Crop box minimum x coordinate\\
  \texttt{CMINY} & Crop box minimum y coordinate\\
  \texttt{CMAXX} & Crop box maximum x coordinate\\
  \texttt{CMAXY} & Crop box maximum y coordinate
\end{tabular}
\end{center}

\noindent For example, we may write \texttt{PMINX} \texttt{PMINY} to stand for the coordinate of the lower left corner of the page.

Simple arithmetic may be performed using the words \texttt{add}, \texttt{sub}, \texttt{mul} and \texttt{div} to stand for addition, subtraction, multiplication and division. For example, one may write \texttt{14in\hspace{-1mm} sub\hspace{-1mm} 30pt} or \texttt{PMINX\hspace{-1mm} mul\hspace{-1mm} 2}

\section{Setting the Producer and Creator}
\index{producer}\index{creator}
The \texttt{-producer} and \texttt{-creator} options may be added to any \texttt{cpdf} command line to set the producer and/or creator of the PDF file. If the file was converted from another format, the \textit{creator} is the program producing the original, the \textit{producer} the program converting it to PDF.

\begin{framed}
  \small\verb!cpdf -merge in.pdf in2.pdf -producer MyMerger -o out.pdf!\\

  \vspace{2.5mm}
  \noindent Merge \texttt{in.pdf} and \texttt{in2.pdf}, setting the producer to \texttt{MyMerger} and writing the output to \texttt{out.pdf}.\end{framed}

\section{PDF Version Numbers}
\index{version number}
When an operation which uses a part of the PDF standard which was introduced in
a later version than that of the input file, the PDF version in the output file
is set to the later version (most PDF viewers will try to load any PDF file,
even if it is marked with a later version number). However, this automatic
version changing may be suppressed with the \texttt{-keep-version} option. If you wish to manually alter the PDF version of a file, use the
\texttt{-set-version} operation described in Section \ref{setversion}.

\section{File IDs}
\index{file ID}
PDF files contain an ID (consisting of two parts), used by some workflow
systems to uniquely identify a file. To change the ID, behavior, use the
\texttt{-change-id} operation. This will create a new ID for the output file. 

\begin{framed}
\small\verb!cpdf -change-id in.pdf -o out.pdf!

\vspace{2.5mm}
\noindent Write \texttt{in.pdf} to \texttt{out.pdf}, changing the ID.
\end{framed}

\section{Linearization}
\index{linearization}
Linearized PDF is a version of the PDF format in which the data is held in a
special manner to allow content to be fetched only when needed. This means
viewing a multipage PDF over a slow connection is more responsive. By default,
\cpdf\ does not linearize output files. To make it do so, add the \texttt{-l}
option to the command line, in addition to any other command being used. For example:

\begin{framed}
\small\verb!cpdf -l in.pdf -o out.pdf!

\vspace{2.5mm}
\noindent Linearize the file \texttt{in.pdf}, writing to \texttt{out.pdf}.
\end{framed}

\noindent This requires the existence of the external program \texttt{cpdflin} which is provided with commercial versions of \texttt{cpdf}. This must be installed as described in the installation documentation provided with your copy of \texttt{cpdf}. If you are unable to install \texttt{cpdflin}, you must use \texttt{-cpdflin} to let \texttt{cpdf} know where to find it:

\begin{framed}
\small\verb!cpdf.exe -cpdflin "C:\\cpdflin.exe" -l in.pdf -o out.pdf!

\vspace{2.5mm}
\noindent Linearize the file \texttt{in.pdf}, writing to \texttt{out.pdf}.
\end{framed}

\noindent In extremis, you may place \texttt{cpdflin} and its resources in the current working directory, though this is not recommended. For further help, refer to the installation instructions for your copy of \texttt{cpdf}.

To keep the existing linearization status of a file (produce linearized output if the input is linearized and the reverse), use \texttt{-keep-l} instead of \texttt{-l}.

\section{Object Streams}
\index{object stream}
PDF 1.5 introduced a new mechanism for storing objects to save space: object streams. by default, \texttt{cpdf} will preserve object streams in input files, creating no more. To prevent the retention of existing object streams, use \texttt{-no-preserve-objstm}:

\begin{framed}
\small\verb!cpdf -no-preserve-objstm in.pdf -o out.pdf!

\vspace{2.5mm}
\noindent Write the file \texttt{in.pdf} to \texttt{out.pdf}, removing any object streams.
\end{framed}

\noindent To create new object streams if none exist, or augment the existing ones, use \texttt{-create-objstm}:

\begin{framed}
\small\verb!cpdf -create-objstm in.pdf -o out.pdf!

\vspace{2.5mm}
\noindent Write the file \texttt{in.pdf} to \texttt{out.pdf}, preserving any existing object streams, and creating any new ones for new objects which have been added.
\end{framed}

\noindent To create wholly new object streams, use both options together:

\begin{framed}
\small\verb!cpdf -create-objstm -no-preserve-objstm in.pdf -o out.pdf!

\vspace{2.5mm}
\noindent Write the file \texttt{in.pdf} to \texttt{out.pdf} with wholly new object streams.
\end{framed}

\noindent Files written with object streams will be set to PDF 1.5 or higher, unless \texttt{-keep-version} is used (see above).

\section{Malformed Files}
\index{malformed file}
\label{fast}
There are many malformed PDF files in existence, including many produced by
otherwise-reputable applications. \cpdf\ attempts to correct these problems
silently.

Grossly malformed files will be reconstructed. The reconstruction
progress is shown on \verb!stderr! (Standard Error):

\begin{framed}
\noindent\small\verb!$cpdf in.pdf -o out.pdf!\\
\small\verb!couldn't lex object number!\\
\small\verb!Attempting to reconstruct the malformed pdf in.pdf...!\\
\small\verb!Read 5530 objects!\\
\small\verb$Malformed PDF reconstruction succeeded!$
\end{framed}

\noindent If \texttt{cpdf} cannot reconstruct a malformed file, it is able to use the \texttt{gs} program to try to reconstruct the PDF file, if you have it installed. For example, if \texttt{gs} is installed and in your path, we might try:

\begin{framed}
\noindent\small\verb!cpdf -gs gs -gs-malformed in.pdf -o out.pdf!\end{framed}

\noindent To suppress the output of \texttt{gs} use the \texttt{-gs-quiet} option.

If the malformity lies inside an individual page of the PDF, rather than in its gross structure, cpdf may appear to succeed in reconstruction, only to fail when processing a page (e.g when adding text). To force the use of \texttt{gs} to pre-process such files so cpdf cannot fail on them, use \texttt{-gs\--malformed\--force}:

\begin{framed}
\noindent\small\verb!cpdf in.pdf -gs gs -gs-malformed-force -o out.pdf [-gs-quiet]!\end{framed}

\noindent The command line for \texttt{-gs-malformed-force} must be of \textit{precisely} this form. Sometimes, on the other hand, we might wish \texttt{cpdf} to fail immediately on any malformed file, rather than try its own reconstruction process. The option \texttt{-error-on-malformed} achieves this.


Sometimes (old, pre-ISO standardisation) files can be technically well-formed but use inefficient PDF
constructs.  If you are sure the input files you are using are
well formed, the \texttt{-fast} option may be added to the command line (or, if
using \texttt{AND}, to each section of the command line). This will use certain
shortcuts which speed up processing, but would fail on badly-produced files. The \verb!-fast! option may be used with:

\begin{framed}
\small\noindent Chapter \ref{pages}\\
\noindent\small\verb!-rotate-contents -upright -vflip -hflip!\\
\small\verb!-shift -scale-page -scale-to-fit -scale-contents!\\
\small\verb!-show-boxes -hard-box -trim-marks!\\

\noindent Chapter \ref{stamps}\\
\noindent\small\verb!-add-text -add-rectangle!\\
\small\verb!-stamp-on -stamp-under -combine-pages!\\

\noindent Chapter \ref{multipage}\\
\noindent\verb!-impose -impose-xy -twoup -twoup-stack!
\end{framed}

\noindent If problems occur, refrain from using \verb!-fast!.

\section{Error Handling}
\index{error handling}
When \cpdf\ encounters an error, it exits with code 2. An error message is
displayed on \texttt{stderr} (Standard Error). In normal usage, this means it's
displayed on the screen. When a bad or inappropriate password is given, the exit code is 1.

\section{Control Files}
\index{control file}
\begin{framed}
  \noindent\small\verb!cpdf -control <filename>!\\
  \noindent\small\verb!cpdf -args <filename>!
\end{framed}

Some operating systems have a limit on the length of a command line. To
circumvent this, or simply for reasons of flexibility, a control file may be
specified from which arguments are drawn. This file does not support the full
syntax of the command line. Commands are separated by whitespace, quotation
marks may be used if an argument contains a space, and the sequence \verb!\"!
may be used to introduce a genuine quotation mark in such an argument.

Several \verb!-control! arguments may be specified, and may be mixed in with
conventional command-line arguments. The commands in each control file are
considered in the order in which they are given, after all conventional
arguments have been processed. It is recommended to use \texttt{-args} in all new applications. However, \texttt{-control} will be supported for legacy applications.

To avoid interference between \texttt{-control} and \texttt{AND}, a new mechanism has been added. Using \texttt{-args} in place of \texttt{-control} will perform direct textual substitution of the file into the command line, prior to any other processing.


\section{String Arguments}
Command lines are handled differently on each operating system. Some
characters are reserved with special meanings, even when they occur inside
quoted string arguments. To avoid this problem, \cpdf\ performs processing on
string arguments as they are read.

A backslash is used to indicate that a character which would otherwise be
treated specially by the command line interpreter is to be treated literally. For
example, Unix-like systems attribute a special meaning to the exclamation mark, so
the command line
\begin{framed}
 \small\verb?cpdf -add-text "Hello!" in.pdf -o out.pdf?
\end{framed}
\noindent would fail. We must escape the exclamation mark with a backslash:
\begin{framed}
 \small\verb?cpdf -add-text "Hello\!" in.pdf -o out.pdf?
\end{framed}
\noindent It follows that backslashes intended to be taken literally must themselves be
escaped (i.e. written \verb!\\!).

\section{Text Encodings}
\index{text!encodings}
\label{textencodings}

Some \texttt{cpdf} commands write text to standard output, or read text from
the command line or configuration files. These are:

\begin{framed}
  \noindent\small\verb!-info!\\
  \noindent\small\verb!-list-bookmarks!\\
  \noindent\small\verb!-set-author! et al.\\
  \noindent\small\verb!-list-annotations!\\
  \noindent\small\verb!-dump-attachments!
\end{framed}

\noindent There are three options to control how the text is interpreted:

\begin{framed}
  \noindent\small\verb!-utf8!\\
  \noindent\small\verb!-stripped!\\
  \noindent\small\verb!-raw!
\end{framed}

\noindent Add \verb!-utf8! to use Unicode UTF8, \verb!-stripped! to convert to 7
bit ASCII by dropping any high characters, or \verb!-raw! to perform no
processing. The default unless specified in the documentation for an individual operation is \verb!-stripped!.

\section{Font Embedding}
\index{font!embedding}
Use the \texttt{-no-embed-font} to avoid embedding the Standard 14 Font metrics when adding text with \texttt{-add-text}.

\begin{cpdflib}
\clearpage
\section*{C Interface}
\begin{small}\tt
\lstinputlisting{splits/c01}
\lstinputlisting{splits/c02}
\end{small}
\end{cpdflib}

\begin{pycpdflib}
\clearpage
\section*{Python Interface}
\begin{small}\tt
\lstinputlisting{pysplits/c01}
\lstinputlisting{pysplits/c02}
\end{small}
\end{pycpdflib}

\begin{dotnetcpdflib}
\clearpage
\section*{.NET Interface}
\begin{small}\tt
\lstinputlisting{dotnetsplits/cm1}
\lstinputlisting{dotnetsplits/c01}
\lstinputlisting{dotnetsplits/c02}
\end{small}
\end{dotnetcpdflib}

\begin{jcpdflib}
\clearpage
\section*{Java Interface}
\begin{small}\tt
\lstinputlisting{javasplits/cm1}
\lstinputlisting{javasplits/c00}
\lstinputlisting{javasplits/c01}
\lstinputlisting{javasplits/c02}
\end{small}
\end{jcpdflib}

\begin{jscpdflib}
\clearpage
\section*{JavaScript Interface}
\begin{small}\tt
\lstinputlisting{javascriptsplits/cm1}
\lstinputlisting{javascriptsplits/c00}
\lstinputlisting{javascriptsplits/c01}
\lstinputlisting{javascriptsplits/c02}
\end{small}
\end{jscpdflib}

\chapter{Merging and Splitting}\label{chap:2}
  \begin{framed}
  \small
  \noindent\begin{verbatim}
cpdf -merge in1.pdf [<range>] in2.pdf [<range>] [<more names/ranges>]
     [-collate] [-retain-numbering] [-remove-duplicate-fonts]
     [-merge-add-bookmarks [-merge-add-bookmarks-use-titles]]
     -o out.pdf\end{verbatim}

  \vspace{1.5mm}
  \noindent\verb!cpdf -split in.pdf -o <format> [-chunk <chunksize>]!

  \vspace{1.5mm}
  \noindent\verb!cpdf -split-bookmarks <level> in.pdf [-utf8] -o <format>!
  \end{framed}

  \vspace{12mm}
  \section{Merging}
\index{merging}
  The \texttt{-merge} operation allow the merging of several files into one.
Ranges can be used to select only a subset of pages from each
input file in the output. The output file consists of the concatenation of all
the input pages in the order specified on the command line. Actually, the
\texttt{-merge} can be omitted, since this is the default operation of \cpdf.

  \begin{framed}\small
    \verb!cpdf -merge a.pdf 1 b.pdf 2-end -o out.pdf!

    \vspace{2.5mm}
    \noindent Take page one of \texttt{a.pdf} and all but the first page of
\texttt{b.pdf}, merge them and produce \texttt{out.pdf}.

    \vspace{1.5mm}
    \verb!cpdf -merge -idir files -o out.pdf!

    \vspace{2.5mm}
    \noindent Merge all files from directory \texttt{files}, producing \texttt{out.pdf}.

  \end{framed}

\noindent Merge maintains bookmarks, named destinations, and name dictionaries.

PDF features which cannot be merged are retained if they are from
the document which first exhibits that feature.

The \texttt{-collate} option collates pages: that is to say, it takes the first page from the first document and its range, then the first page from the second document and its range and so on. When all first pages have been taken, it begins on the second from each range, and so on.\index{pages!collate}\index{collation}

The \texttt{-retain-numbering} option keeps the PDF page numbering labels of
each document intact, rather than renumbering the output pages from 1.

The \texttt{-remove-duplicate-fonts} option ensures that fonts used in more than one
of the inputs only appear once in the output.

The \texttt{-merge-add-bookmarks} option adds a top-level bookmark for each file, using the filename. Any existing bookmarks are retained. The \texttt{-merge\--add\--bookmarks\--use\--titles}, when used in conjunction with \texttt{-merge-add-bookmarks}, will use the title from each PDF's metadata instead of the filename.

  \section{Splitting}
\index{splitting}
   The \texttt{-split} operation splits a PDF file into a number of parts which
are written to file, their names being generated from a \emph{format}. The
optional \texttt{-chunk} option allows the number of pages written to each
output file to be set. 
  \begin{framed}\small
    \verb!cpdf -split a.pdf -o out%%%.pdf!

    \vspace{2.5mm}
    \noindent Split \texttt{a.pdf} to the files \texttt{out001.pdf}, \texttt{out002.pdf} etc.

    \vspace{2.5mm}
    \verb!cpdf a.pdf even AND -split -chunk 10 -o dir/out%%%.pdf!

    \vspace{2.5mm}
    \noindent Split the even pages of \texttt{a.pdf} to the files
\texttt{out001.pdf}, \texttt{out002.pdf} etc. with at most ten pages in each
file. The directory (folder) \texttt{dir} must exist.
  \end{framed}
\noindent If the output format does not provide enough numbers for the files generated,
the result is unspecified. The following format operators may be used:

\begin{center}
\begin{tabular}{rl}
  \verb!%, %%, %%% etc.! & Sequence number padded to the number of percent signs\\
  \texttt{@F} & Original filename without extension \\
  \texttt{@N} & Sequence number without padding zeroes \\
  \texttt{@S} & Start page of this chunk \\
  \texttt{@E} & End page of this chunk \\
  \texttt{@B} & Bookmark name at this page \\
\end{tabular}
\end{center}

\noindent Numbers padded to a fixed width field by zeroes may be obtained for \texttt{@S} and \texttt{@E} by following them with more \texttt{@} signs e.g \texttt{@E@@@} for a fixed width of three.

  \section{Splitting on Bookmarks}
  \index{splitting!on bookmarks}
  The \texttt{-split-bookmarks <level>} operation splits a PDF file into a number of
parts, according to the page ranges implied by the document's bookmarks. These
parts are then written to file with names generated from the given format.
  
Level 0 denotes the top-level bookmarks, level 1 the next level (sub-bookmarks)
and so on. So \texttt{-split-bookmarks 1} creates breaks on level 0 and level
1 boundaries.

  \begin{framed}\small
    \verb!cpdf -split-bookmarks 0 a.pdf -o out%%%.pdf!

    \vspace{2.5mm}
    \noindent Split \texttt{a.pdf} to the files \texttt{out001.pdf},
\texttt{out002.pdf} on bookmark boundaries.

  \end{framed}
\noindent Now, there may be many bookmarks on a single page (for instance, if
paragraphs are bookmarked or there are two subsections on one page). The splits
calculated by \texttt{-split-bookmarks} ensure that each page appears in only
one of the output files.
  It is possible to use the \texttt{@} operators above, including operator \texttt{@B} which expands to the text of the bookmark:

  \begin{framed}\small
    \verb!cpdf -split-bookmarks 0 a.pdf -o @B.pdf!

    \vspace{2.5mm}
    \noindent Split \texttt{a.pdf} on bookmark boundaries, using the bookmark text as the filename.

  \end{framed}
\noindent The bookmark text used for a name is converted from unicode to 7 bit ASCII, and the following characters are removed, in addition to any character with ASCII code less than 32:
  \begin{framed}
  \centering
  \verb! / ? < > \ : * | " ^ + =!
  \end{framed}

  To prevent this process, and convert bookmark names to UTF8 instead, add \texttt{-utf8} to the command.

\section{Encrypting with Split and Split Bookmarks}
\index{encryption}
The encryption parameters described in Chapter \ref{encryption} may be added to the command line to encrypt each split PDF. Similarly, the \texttt{-recrypt} switch described in Chapter \ref{basicusage} may by given to re-encrypt each file with the existing encryption of the source PDF. 
\pagestyle{empty}\thispagestyle{fancy}

\begin{cpdflib}
\clearpage
\section*{C Interface}
\begin{small}\tt
\lstinputlisting{splits/c03}
\end{small}
\end{cpdflib}

\begin{pycpdflib}
\clearpage
\section*{Python Interface}
\begin{small}\tt
\lstinputlisting{pysplits/c03}
\end{small}
\end{pycpdflib}

\begin{dotnetcpdflib}
\clearpage
\section*{.NET Interface}
\begin{small}\tt
\lstinputlisting{dotnetsplits/c03}
\end{small}
\end{dotnetcpdflib}

\begin{jcpdflib}
\clearpage
\section*{Java Interface}
\begin{small}\tt
\lstinputlisting{javasplits/c03}
\end{small}
\end{jcpdflib}

\begin{jscpdflib}
\clearpage
\section*{JavaScript Interface}
\begin{small}\tt
\lstinputlisting{javascriptsplits/c03}
\end{small}
\end{jscpdflib}

\chapter{Pages}\label{chap:3}
\pagestyle{fancy}
  \label{pages}
  \begin{framed}
  \small\noindent\verb!cpdf -scale-page "<scale x> <scale y>" [-fast] in.pdf [<range>] -o out.pdf!
   
  \vspace{1.5mm}
  \small\noindent\verb!cpdf -scale-to-fit "<x size> <y size>" [-fast]!\\
        \noindent\verb!     [-scale-to-fit-scale <scale>]!\\    
        \noindent\verb!     in.pdf [<range>] -o out.pdf!

  \vspace{1.5mm}
  \small\noindent\verb!cpdf -scale-contents [<scale>] [<position>] [-fast]!\\
        \noindent\verb!     in.pdf [<range>] -o out.pdf!
  
  \vspace{1.5mm}
  \small\noindent\verb!cpdf -shift "<shift x> <shift y>" [-fast] in.pdf [<range>] -o out.pdf!

  \vspace{1.5mm}
  \small\noindent\verb!cpdf -rotate <angle> in.pdf [<range>] -o out.pdf!

  \vspace{1.5mm}
  \small\noindent\verb!cpdf -rotateby <angle> in.pdf [<range>] -o out.pdf!

  \vspace{1.5mm}
  \small\noindent\verb!cpdf -upright [-fast] in.pdf [<range>] -o out.pdf!

  \vspace{1.5mm}
  \small\noindent\verb!cpdf -rotate-contents <angle> [-fast] in.pdf [<range>] -o out.pdf!


  \vspace{1.5mm}
  \small\noindent\verb!cpdf -hflip [-fast] in.pdf [<range>] -o out.pdf!

  \vspace{1.5mm}
  \small\noindent\verb!cpdf -vflip [-fast] in.pdf [<range>] -o out.pdf!
  
  \vspace{1.5mm}
  \small\noindent\verb!cpdf -mediabox "<x> <y> <w> <h>" in.pdf [<range>] -o out.pdf!


  \vspace{1.5mm}
  \small\noindent\verb!cpdf -cropbox "<x> <y> <w> <h>" in.pdf [<range>] -o out.pdf!

  \vspace{1.5mm}
  \small\noindent\verb!cpdf -remove-cropbox in.pdf [<range>] -o out.pdf!

  \vspace{1.5mm}
  (Also \texttt{bleed}, \texttt{art}, and \texttt{trim} versions of these two commands, for example  \texttt{-artbox}, \texttt{-remove-trimbox})

  \vspace{1.5mm}
  \small\noindent\verb!cpdf -frombox <boxname> -tobox <boxname> [-mediabox-if-missing]! \\
  \noindent\verb!     in.pdf [<range>] -o out.pdf!

  \vspace{1.5mm}
  \small\noindent\verb!cpdf -hard-box <boxname> [-fast] in.pdf [<range>]!\\
  \small\noindent\verb!     [-mediabox-if-missing] -o out.pdf!

  \vspace{1.5mm}
  \small\noindent\verb!cpdf -show-boxes [-fast] in.pdf [<range>] -o out.pdf!

  \vspace{1.5mm}
  \small\noindent\verb!cpdf -trim-marks [-fast] in.pdf [<range>] -o out.pdf!

  \end{framed}

  \section{Page Sizes}
  \index{page!size}
\label{papersizes}
  Any time when a page size is required, instead of writing, for instance \texttt{"210mm 197mm"} one can instead write \texttt{a4portrait}. Here is a list of supported page sizes:

{\small
  \smallgap
  \begin{tabular}{lll}
  \texttt{a0portrait} & \texttt{a1portrait} & \texttt{a2portrait} \\
  \texttt{a3portrait} & \texttt{a4portrait} & \texttt{a5portrait} \\
  \texttt{a6portrait} & \texttt{a7portrait} & \texttt{a8portrait} \\
  \texttt{a9portrait} & \texttt{a10portrait} & \\
  \\
  \texttt{a0landscape} & \texttt{a1landscape} & \texttt{a2landscape} \\
  \texttt{a3landscape} & \texttt{a4landscape} & \texttt{a5landscape} \\
  \texttt{a6landscape} & \texttt{a7landscape} & \texttt{a8landscape} \\
  \texttt{a9landscape} & \texttt{a10landscape} & \\
  \\
  \texttt{usletterportrait} & \texttt{usletterlandscape} & \\
  \texttt{uslegalportrait} & \texttt{uslegallandscape} &
  \end{tabular}
}

Note that this also works when four numbers are required: for example, when setting the mediabox "0 0 a3landscape" will suffice.

  \section{Scale Pages}
\index{scale pages}
  The \texttt{-scale-page} operation scales each page in the range by the X and
Y factors given. This scales both the page contents, and the page size itself. It also scales any Crop Box and other boxes (Art Box, Trim Box etc). As with several of these commands, remember to take into account any page rotation when considering what the X and Y axes relate to.

  \begin{framed}
  \small\noindent\verb!cpdf -scale-page "2 2" in.pdf -o out.pdf!

  \vspace{2.5mm}
  \noindent Convert an A4 page to A2, for instance.
  \end{framed}

  \noindent The \texttt{-scale-to-fit} operation scales each page in the range to fit a
  given page size, preserving aspect ratio and centering the result.

  \begin{framed}
  \small\noindent\verb!cpdf -scale-to-fit "297mm 210mm" in.pdf -o out.pdf!
  \small\noindent\verb!cpdf -scale-to-fit a4portrait in.pdf -o out.pdf!

  \vspace{2.5mm}
  \noindent Scale a file's pages to fit A4 portrait.
  \end{framed}
  %The \texttt{-scale-to-fit-best} and \texttt{-scale-to-fit-minus} are similar, but will rotate a page by $90^\circ$ or $-90^\circ$ respectively on any page where doing so would maximise the scale.

  \noindent The scale can optionally be set to a percentage of the available area, instead of filling it.
  \begin{framed}
  \small\noindent\verb!cpdf -scale-to-fit a4portrait -scale-to-fit-scale 0.9 in.pdf -o out.pdf!


  \vspace{2.5mm}
  \noindent Scale a file's pages to fit A4 portrait, scaling the page 90\% of its possible size.
  \end{framed}

\noindent NB: \texttt{-scale-to-fit} operates with respect to the media box not the crop box. If necessary, set the media box to be equal to the crop box first. In addition, \texttt{-scale-to-fit} presently requires that the origin of the media box be (0, 0). This can be assured by preprocessing with \texttt{-upright} (described elsewhere in this chapter).

The \texttt{-scale-contents} operation scales the contents about the center
  of the crop box (or, if absent, the media box), leaving the page dimensions
  (boxes) unchanged.

  \begin{framed}
  \small\noindent\verb!cpdf -scale-contents 0.5 in.pdf -o out.pdf!

  \vspace{2.5mm}
  \noindent Scale a file's contents on all pages to 50\% of its original dimensions.
  \end{framed}

  \noindent To scale about a point other than the center, one can use the positioning commands described in Section \ref{position}. For example:
  
  \begin{framed}
  \small\noindent\verb!cpdf -scale-contents 0.5 -topright 20 in.pdf -o out.pdf!

  \vspace{2.5mm}
  \noindent Scale a file's contents on all pages to 50\% of its original dimensions about a point 20pts from its top right corner.
  \end{framed}

  

  \section{Shift Page Contents}
  \index{shift page contents}

  The \texttt{-shift} operation shifts the contents of each page in the range
by X points horizontally and Y points vertically.

  \begin{framed}
  \small\noindent\verb!cpdf -shift "50 0" in.pdf even -o out.pdf!

  \vspace{2.5mm}

  \noindent Shift pages to the right by 50 points (for instance, to increase
the binding margin).

  \end{framed}
  \section{Rotating Pages}
\index{rotate!pages}

There are two ways of rotating pages: (1)~setting a value in the PDF file which
asks the viewer (e.g. Acrobat) to rotate the page on-the-fly when viewing it
(use \texttt{-rotate} or \texttt{-rotateby}) and (2)~actually rotating the page
contents and/or the page dimensions (use \texttt{-upright} (described elsewhere in this chapter) afterwards or
\texttt{-rotate-contents} to just rotate the page contents).

  The possible values for \texttt{-rotate} and \texttt{-rotate-by} are 0, 90,
180 and 270, all interpreted as being clockwise. Any value may be used for
\texttt{-rotate-contents}.
  
The \texttt{-rotate} operation sets the viewing rotation of the selected pages to
the absolute value given.
  \begin{framed}
  \small\verb!cpdf -rotate 90 in.pdf -o out.pdf!

  \vspace{2.5mm}
  \noindent Set the rotation of all the pages in the input file to ninety degrees clockwise.
  \end{framed}
  \noindent The \texttt{-rotateby} operation changes the viewing rotation of all the
given pages by the relative value given.
  \begin{framed}
  \small\verb!cpdf -rotateby 90 in.pdf -o out.pdf!

  \vspace{2.5mm}
  \noindent Rotate all the pages in the input file by ninety degrees clockwise.
  \end{framed}
  \noindent The \texttt{-rotate-contents} operation rotates the contents and dimensions
of the page by the given relative value.
  \index{rotate!contents}
  \begin{framed}
  \small\verb!cpdf -rotate-contents 90 in.pdf -o out.pdf!

  \vspace{2.5mm}

  \noindent Rotate all the page contents in the input file by
ninety degrees clockwise. Does not change the page dimensions.
  \end{framed}

  \label{upright}
   \noindent The \texttt{-upright} operation does whatever combination of
\texttt{-rotate} and \texttt{-rotate-contents} is required to change the
rotation of the document to zero without altering its appearance. In addition, it makes sure the media box has its origin at (0,0), changing other boxes to compensate. This is important because some operations in CPDF (such as scale-to-fit), and in other PDF-processing programs, work properly only when the origin is (0, 0).

  \begin{framed}
  \small\verb!cpdf -upright in.pdf -o out.pdf!

  \vspace{2.5mm}

  \noindent Make pages upright.
  \end{framed}

  \section{Flipping Pages}
\index{flip pages}
  The \texttt{-hflip} and \texttt{-vflip} operations flip the contents of the
chosen pages horizontally or vertically. No account is taken of the current
page rotation when considering what "horizontally" and "vertically" mean, so you may like to use \texttt{-upright} (see above) first.
  \begin{framed}
    \small\verb!cpdf -hflip in.pdf even -o out.pdf!

    \vspace{2.5mm}
    \noindent Flip the even pages in \texttt{in.pdf} horizontally.

    \vspace{2.5mm}
    \verb!cpdf -vflip in.pdf -o out.pdf!

    \vspace{2.5mm}
    \noindent Flip all the pages in \texttt{in.pdf} vertically.
  \end{framed}

  \section{Boxes and Cropping}
 \index{crop pages}
\index{media box}
  All PDF files contain a \textit{media box} for each page, giving the
dimensions of the paper. To change these dimensions (without altering the page
contents in any way), use the \texttt{-mediabox} operation.
  \begin{framed}
  \small\verb!cpdf -mediabox "0pt 0pt 500pt 500pt" in.pdf -o out.pdf!

  \vspace{2.5mm}
  \noindent Set the media box to 500 points square.
  \end{framed}
  \noindent The four numbers are minimum x, minimum y, width, height. x
coordinates increase to the right, y coordinates increase upwards.
  PDF file can also optionally contain a \textit{crop box} for each page,
defining to what extent the page is cropped before being displayed or printed.
A crop box can be set, changed and removed, without affecting the underlying
media box. To set or change the crop box use \texttt{-cropbox}. To remove any
existing crop box, use \texttt{-remove-cropbox}.
  \begin{framed}
  \small\verb!cpdf -cropbox "0pt 0pt 200mm 200mm" in.pdf -o out.pdf!

  \vspace{2.5mm}
  \noindent Crop pages to the bottom left 200-millimeter square of the page.

  \vspace{2.5mm}
  \verb!cpdf -remove-cropbox in.pdf -o out.pdf!
  
  \vspace{2.5mm}
  \noindent Remove cropping.
  \end{framed}

\noindent Note that the crop box is only obeyed in some viewers. Similar operations are available for the bleed, art, and trim boxes (\texttt{-art}, \texttt{-remove-bleed} etc.)

  \begin{framed}
  \small\noindent\verb!cpdf -frombox <boxname> -tobox <boxname> [-mediabox-if-missing]! \\
  \noindent\verb!     in.pdf [<range>] -o out.pdf!

  \vspace{2.5mm}
  \noindent Copy the contents of one box to another.

  \end{framed}
  \noindent This operation copies the contents of one box (Media box, Crop box, Trim box etc.) to another. If \texttt{-mediabox-if-missing} is added, the media box will be substituted when the 'from' box is not set for a given page. For example

  \begin{framed}
    \small\verb!cpdf -frombox /TrimBox -tobox /CropBox in.pdf -o out.pdf!
  \end{framed}
  \noindent copies the Trim Box of each page to the Crop Box of each page. The possible boxes are \texttt{/MediaBox}, \texttt{/CropBox}, \texttt{/BleedBox}, \texttt{/TrimBox}, \texttt{/ArtBox}.\pagestyle{empty}\thispagestyle{fancy}

A hard box (one which clips its contents by inserting a clipping rectangle) may be created with the \texttt{-hard-box} operation:

  \begin{framed}
    \small\verb!cpdf -hard-box /TrimBox in.pdf -o out.pdf!
  \end{framed}

\noindent This means the resultant file may be used as a stamp without contents outside the given box reappearing. The \texttt{-mediabox-if-missing} option may also be used here.

\section{Showing Boxes and Printer's Marks}
\index{printer's marks}\index{trim marks}

The \texttt{-show-boxes} operation displays the boxes present on each page as method of debugging. Since boxes may be coincident, they are shown in differing colours and dash patterns so they may be identified even where they overlap. The colours are:

\medskip
\begin{tabular}{ll}
Media box & Red \\
Crop box & Green \\
Art box & Blue \\
Trim box & Orange \\
Bleed box & Pink 
\end{tabular}
\medskip

\pagestyle{fancy}The \texttt{-trim-marks} operation adds trim marks to a PDF file. The trim box must be present.

\begin{cpdflib}
\clearpage
\section*{C Interface}
\begin{small}\tt
\lstinputlisting{splits/c04}
\end{small}
\end{cpdflib}

\begin{pycpdflib}
\clearpage
\section*{Python Interface}
\begin{small}\tt
\lstinputlisting{pysplits/c04}
\end{small}
\end{pycpdflib}

\begin{dotnetcpdflib}
\clearpage
\section*{.NET Interface}
\begin{small}\tt
\lstinputlisting{dotnetsplits/c04}
\end{small}
\end{dotnetcpdflib}

\begin{jcpdflib}
\clearpage
\section*{Java Interface}
\begin{small}\tt
\lstinputlisting{javasplits/c04}
\end{small}
\end{jcpdflib}

\begin{jscpdflib}
\clearpage
\section*{JavaScript Interface}
\begin{small}\tt
\lstinputlisting{javascriptsplits/c04}
\end{small}
\end{jscpdflib}

\chapter{Encryption and Decryption}\label{chap:4}
\pagestyle{fancy}
\label{encryption}
\index{encryption}
\index{decryption}
  \begin{framed}
    \small\noindent\verb!cpdf -encrypt <method> [-pw=]<owner> [-pw=]<user>!\\
    \noindent\verb!     [-no-encrypt-metadata] <permissions> in.pdf -o out.pdf!

    \vspace{1.5mm}
    \noindent\verb!cpdf -decrypt [-decrypt-force] in.pdf owner=<owner password> -o out.pdf!
  \end{framed}
  \label{crypt}
  \section{Introduction}
  PDF files can be encrypted using various types of encryption and attaching
various permissions describing what someone can do with a particular document
(for instance, printing it or extracting content). There are two types of
person:
  \begin{description}
    \item The \textbf{User} can do to the document what is allowed in the permissions.
    \item The \textbf{Owner} can do anything, including altering the permissions or removing encryption entirely.
  \end{description}
  There are five kinds of encryption:
  \begin{itemize}
  \item 40-bit encryption (method \texttt{40bit}) in Acrobat 3 (PDF 1.1) and above
  \item 128-bit encryption (method \texttt{128bit}) in Acrobat 5 (PDF 1.4) and above
  \item 128-bit AES encryption (method \texttt{AES}) in Acrobat 7 (PDF 1.6) and above
  \item 256-bit AES encryption (method \texttt{AES256}) in Acrobat 9 (PDF 1.7) -- \textit{this is deprecated -- do not use for new documents}
  \item 256-bit AES encryption (method \texttt{AES256ISO}) in PDF 2.0
  \end{itemize}

   \vspace{2mm}
   \noindent All encryption supports these kinds of permissions:

   \vspace{2mm}
   \begin{tabular}{ll}
     \texttt{-no-edit} & Cannot change the document\\
     \texttt{-no-print} & Cannot print the document\\
     \texttt{-no-copy} & Cannot select or copy text or graphics\\
     \texttt{-no-annot} & Cannot add or change form fields or annotations\\
   \end{tabular}

   \vspace{2mm}
   \noindent In addition, 128-bit encryption (Acrobat 5 and above) and AES encryption supports these:

   \vspace{2mm}
   \begin{tabular}{ll}
     \texttt{-no-forms} & Cannot edit form fields\\
     \texttt{-no-extract} & Cannot extract text or graphics\\
     \texttt{-no-assemble} & Cannot merge files etc.\\
     \texttt{-no-hq-print} & Cannot print high-quality\\
   \end{tabular}

  \vspace{2mm}
  \noindent Add these options to the command line to prevent each operation.

  \vspace{2mm}

  \section{Encrypting a Document}
  To encrypt a document, the owner and user passwords must be given (here, \texttt{fred} and \texttt{charles} respectively):
  \begin{framed}
    \small\verb!cpdf -encrypt 40bit fred charles -no-print in.pdf -o out.pdf!

    \vspace{1.5mm}
    \small\verb!cpdf -encrypt 128bit fred charles -no-extract in.pdf -o out.pdf!

    \vspace{1.5mm}
    \small\verb!cpdf -encrypt AES fred "" -no-edit -no-copy in.pdf -o out.pdf!
  \end{framed}
  \noindent A blank user password is
common. In this event, PDF viewers will typically not prompt for a
password for when opening the file or for operations allowable with the user password.
  \begin{framed}
    \vspace{1.5mm}
    \small\verb!cpdf -encrypt AES256 fred "" -no-forms in.pdf -o out.pdf!
  \end{framed}
\noindent In addition, the usual method can be used to give the existing owner
password, if the document is already encrypted.

The optional \texttt{-pw=} preface may be given where a password might begin with a \texttt{-} and thus be confused with a command line option.

When using AES encryption, the option is available to refrain from encrypting the
metadata. Add \texttt{-no-encrypt-metadata} to the command line.

  \section{Decrypting a Document}
  To decrypt a document, the owner password is provided.
  \begin{framed}
    \small\verb!cpdf -decrypt in.pdf owner=fred -o out.pdf!
  \end{framed}
  \noindent The user password cannot decrypt a file.

When appropriate passwords are not available, the option \texttt{-decrypt-force} may be added to the command line to process the file regardless.


\begin{cpdflib}
\clearpage
\section*{C Interface}
\begin{small}\tt
\lstinputlisting{splits/c05}
\end{small}
\end{cpdflib}

\begin{pycpdflib}
\clearpage
\section*{Python Interface}
\begin{small}\tt
\lstinputlisting{pysplits/c05}
\end{small}
\end{pycpdflib}

\begin{dotnetcpdflib}
\clearpage
\section*{.NET Interface}
\begin{small}\tt
\lstinputlisting{dotnetsplits/c05}
\end{small}
\end{dotnetcpdflib}

\begin{jcpdflib}
\clearpage
\section*{Java Interface}
\begin{small}\tt
\lstinputlisting{javasplits/c05}
\end{small}
\end{jcpdflib}

\begin{jscpdflib}
\clearpage
\section*{JavaScript Interface}
\begin{small}\tt
\lstinputlisting{javascriptsplits/c05}
\end{small}
\end{jscpdflib}

\chapter{Compression}\label{chap:5}
  \begin{framed}
     \small\noindent\verb!cpdf -decompress in.pdf -o out.pdf!

     \vspace{1.5mm}
     \noindent\verb!cpdf -compress in.pdf -o out.pdf!
     
     \vspace{1.5mm}
     \noindent\verb!cpdf -squeeze in.pdf [-squeeze-log-to <filename>]!\\
     \noindent\verb!     [-squeeze-no-recompress] [-squeeze-no-pagedata] -o out.pdf!   
   \end{framed}
  \cpdf\ provides basic facilities for decompressing and compressing PDF streams, and for reprocessing the whole file to `squeeze' it.
  \section{Decompressing a Document}
\index{decompressing}
  To decompress the streams in a PDF file, for instance to manually inspect the
PDF, use:
  \begin{framed}
   \small\verb!cpdf -decompress in.pdf -o out.pdf!
  \end{framed}
  \noindent If \cpdf\ finds a compression type it can't cope with, the stream is left compressed. When using \texttt{-decompress}, object streams are not compressed. It may be easier for manual inspection to also remove object streams, by adding the \texttt{-no-preserve-objstm} option to the command.
  \section{Compressing a Document}
\index{compressing}
  To compress the streams in a PDF file, use:
  \begin{framed}
    \small\verb!cpdf -compress in.pdf -o out.pdf!
  \end{framed}
  \noindent\cpdf\ compresses any streams which have no compression using the
  \textbf{Flate\-Decode} method, with the exception of Metadata streams, which
  are left uncompressed.
  
  \section{Squeezing a Document}
\index{squeeze}
  To \textit{squeeze} a PDF file, reducing its size by an average of about twenty percent (though sometimes not at all), use:
  \begin{framed}
    \small\verb!cpdf -squeeze in.pdf -o out.pdf!
  \end{framed}
  \noindent Adding \texttt{-squeeze} to the command line when using another operation will \textit{squeeze} the file or files upon output.
  
  The \texttt{-squeeze} operation writes some information about the squeezing process to standard output. The squeezing process involves several processes which losslessly attempt to reduce the file size. It is slow, so should not be used without thought.

\begin{verbatim}
$ ./cpdf -squeeze in.pdf -o out.pdf
Initial file size is 238169 bytes
Beginning squeeze: 123847 objects
Squeezing... Down to 114860 objects
Squeezing... Down to 114842 objects
Squeezing page data
Recompressing document
Final file size is 187200 bytes,  78.60% of original.
\end{verbatim}

\noindent The \texttt{-squeeze-log-to <filename>} option writes the log to the given file instead of to standard output. Log contents is appended to the end of the log file, preserving existing contents.

There are two options which turn off parts of the squeezer. They are \texttt{-squeeze-no-recompress} for avoiding the reprocessing of compressed sections (especially useful if they are malformed), and \texttt{-squeeze-no-pagedata} for avoiding the reprocessing of page data (ditto). These two options also make the process much faster at the cost of a little less compression. Experiment.

\begin{cpdflib}
\clearpage
\section*{C Interface}
\begin{small}\tt
\lstinputlisting{splits/c06}
\end{small}
\end{cpdflib}

\begin{pycpdflib}
\clearpage
\section*{Python Interface}
\begin{small}\tt
\lstinputlisting{pysplits/c06}
\end{small}
\end{pycpdflib}

\begin{jcpdflib}
\clearpage
\section*{Java Interface}
\begin{small}\tt
\lstinputlisting{javasplits/c06}
\end{small}
\end{jcpdflib}

\begin{jscpdflib}
\clearpage
\section*{JavaScript Interface}
\begin{small}\tt
\lstinputlisting{javascriptsplits/c06}
\end{small}
\end{jscpdflib}

\chapter{Bookmarks}\label{chap:6}
  \begin{framed}
  \small\noindent\verb!cpdf -list-bookmarks [-utf8 | -raw] in.pdf!

  \vspace{1.5mm}
  \small\noindent\verb!cpdf -list-bookmarks-json in.pdf!

  \vspace{1.5mm}
  \small\noindent\verb!cpdf -remove-bookmarks in.pdf -o out.pdf!

  \vspace{1.5mm}
  \small\noindent\verb!cpdf -add-bookmarks <bookmark file> in.pdf -o out.pdf!

  \vspace{1.5mm}
  \small\noindent\verb!cpdf -add-bookmarks-json <bookmark file> in.pdf -o out.pdf!

  \vspace{1.5mm}
  \small\noindent\verb!cpdf -bookmarks-open-to-level <n> in.pdf -o out.pdf!

  \vspace{1.5mm}
  \small\noindent\verb!cpdf -table-of-contents [-toc-title] [-toc-no-bookmark]!\\
  \small\noindent\verb!     [-font <font>] [-font-size <size>] in.pdf -o out.pdf!

  \end{framed}
\index{bookmarks}\index{JSON!add bookmarks from}
\index{document outline}
  PDF bookmarks (properly called the \textit{document outline}) represent a tree
of references to parts of the file, typically displayed at the side of the
screen. The user can click on one to move to the specified place. \cpdf\ provides
facilities to list, add, and remove bookmarks. The format used by the list and
add operations is the same, so you can feed the output of one into the other,
for instance to copy bookmarks.

  \section{List Bookmarks}
\index{bookmarks!listing}\index{JSON!list bookmarks as}
  The \texttt{-list-bookmarks} operation prints (to standard output) the
bookmarks in a file. The first column gives the level of the tree at which a
particular bookmark is. Then the text of the bookmark in quotes. Then the page
number which the bookmark points to. Then (optionally) the word "open" if the
bookmark should have its children (at the level immediately below) visible when
the file is loaded. Then the destination (see below). For example, upon executing
\begin{framed}
  \small\verb!cpdf -list-bookmarks doc.pdf!
\end{framed}

\noindent the result might be:
\begin{framed}{\small\begin{verbatim}
0 "Part 1" 1 open
1 "Part 1A" 2 "[2 /XYZ 200 400 null]"
1 "Part 1B" 3
0 "Part 2" 4
1 "Part 2a" 5\end{verbatim}}\end{framed}
\noindent If the page number is 0, it indicates that clicking on that entry doesn't move to a page.

By default, \cpdf\ converts unicode to ASCII text, dropping characters outside
the ASCII range. To prevent this, and return unicode UTF8 output, add the
\texttt{-utf8} option to the command. To prevent any processing, use the
\texttt{-raw} option. See Section \ref{textencodings} for more information. A newline in a bookmark is represented as \texttt{"\textbackslash n"}.

By using \texttt{-list-bookmarks-json} instead, the bookmarks are formatted as a JSON array, in order, of dictionaries formatted thus:

\begin{verbatim}
{ "level": 0,
  "text": "1 Basic Usage",
  "page": 17,
  "open": false,
  "target":
    [ { "I": 17 },
      { "N": "/XYZ" },
      { "F": 85.039 },
      { "F": 609.307 },
      null ]
}
\end{verbatim}

See chapter 15 for more details of cpdf's JSON formatting. Bookmark text in JSON bookmarks, however, is in UTF8 for ease of use.

\subsection{Destinations}

The destination is an extended description of where the bookmark should point to (i.e it can be more detailed than just giving the page). For example, it may point to a section heading halfway down a page. Here are the possibilities:

\medskip

\begin{tabular}{lp{8cm}}
Format & Description\\\hline
{[\textit{p} /XYZ \textit{left} \textit{top} \textit{zoom}]} & Display page number \textit{p} with (\textit{left}, \textit{top}) positioned at upper-left of window and magnification of \textit{zoom}. Writing ``null'' for any of \textit{left}, \textit{top} or \textit{zoom} specifies no change. A \textit{zoom} of 0 is the same as ``null''.\\
{[\textit{p} /Fit]} & Display page number \textit{p} so as to fit fully within the window.\\
{[\textit{p} /FitH \textit{top}]} & Display page number \textit{p} with vertical coordinate \textit{top} at the top of the window and the page magnified so its width fits the window. A null value for \textit{top} implies no change.\\
{[\textit{p} /FitV \textit{left}]} & Display page number \textit{p} with horizontal coordinate \textit{left} at the left of the window, and the page magnified so its height fits the window. A null value for \textit{left} implies no change. \\
{[\textit{p} /FitR \textit{left} \textit{bottom} \textit{right} \textit{top}]} & Display page number \textit{p} magnified so as to fit entirely within the rectangle specified by the other parameters. \\
{[\textit{p} /FitB]} & As for /Fit but with the page's bounding box (see below).\\
{[\textit{p} /FitBH \textit{top}]} & As for /FitH but with the page's bounding box (see below).\\
{[\textit{p} /FitBV \textit{left}]} & As for /FitV but with the page's bounding box (see below).
\end{tabular}

\medskip

\noindent The \textit{bounding box} is the intersection of the page's crop box and the bounding box of the page contents. Some other kinds of destination may be produced by \texttt{-list-bookmarks}. They will be preserved by \texttt{-add-bookmarks} and may be edited as your risk.


  \section{Remove Bookmarks}
  \label{removebookmarks}
\index{bookmarks!removing}
  The \texttt{-remove-bookmarks} operations removes all bookmarks from the file.
  \begin{framed}
    \small\verb!cpdf -remove-bookmarks in.pdf -o out.pdf!
  \end{framed}

  \section{Add Bookmarks}
  
\index{bookmarks!adding}
  The \texttt{-add-bookmarks} file adds bookmarks as specified by a
\textit{bookmarks file}, a text file in ASCII or UTF8 encoding and in the same format as that produced by the
\texttt{-list-bookmarks} operation. If there are any bookmarks in the input PDF
already, they are discarded. For example, if the file \texttt{bookmarks.txt}
contains the output from \texttt{-list-bookmarks} above, then the command
  \begin{framed}
   \small\verb!cpdf -add-bookmarks bookmarks.txt in.pdf -o out.pdf!
  \end{framed}
\noindent adds the bookmarks to the input file, writing to \texttt{out.pdf}. An error
will be given if the bookmarks file is not in the correct form (in particular,
the numbers in the first column which specify the level must form a proper
tree with no entry being more than one greater than the last).

Bookmarks in JSON format (see above) may be added with \texttt{-add-bookmarks-json}:

  \begin{framed}
   \small\verb!cpdf -add-bookmarks-json bookmarks.json in.pdf -o out.pdf!
  \end{framed}

Remember that strings in JSON bookmark files are in UTF8, rather than as native PDF strings.

\section{Opening bookmarks}
\index{bookmarks!opening at level}
As an alternative to extracting a bookmark file and manipulating the open-status of bookmarks, mass manipulation may be achieved by the following operation:

  \begin{framed}
   \small\verb!cpdf -bookmarks-open-to-level <level> in.pdf -o out.pdf!
  \end{framed}

\noindent A level of 0 will close all bookmarks, level 1 will open just the top level, closing all others etc. To open all of them, pick a sufficiently large level.


\section{Making a Table of Contents}

Cpdf can automatically generate a table of contents from existing bookmarks, adding it to the beginning of the document.

  \begin{framed}
   \small\verb!cpdf -table-of-contents in.pdf -o out.pdf!
  \end{framed}

The page(s) added will have the same dimensions, media and crop boxes as the first page of the original file. The default title is ``Table of Contents'', though this may be changed:

  \begin{framed}
   \small\verb!cpdf -table-of-contents -toc-title "Contents" in.pdf -o out.pdf!
  \end{framed}

An empty title removes the title. The sequence \texttt{\textbackslash n} may be used to split the title into lines. The default font is 12pt Times Roman (and 24pt for the title). The base font and size may be changed with \texttt{-font} and \texttt{-font-size} (see chapter 8 for full details):

  \begin{framed}
   \small\verb!cpdf -table-of-contents -font "Courier-Bold" -font-size 8!\\
   \small\verb!        in.pdf -o out.pdf!
  \end{framed}

By default, an entry for the new table of contents will be added to the document's bookmarks. To suppress this behaviour, add \texttt{-toc-no-bookmark}:

  \begin{framed}
   \small\verb!cpdf -table-of-contents -toc-no-bookmark in.pdf -o out.pdf!
  \end{framed}



\clearpage\pagestyle{empty}

\begin{cpdflib}
\clearpage
\section*{C Interface}
\begin{small}\tt
\lstinputlisting{splits/c07}
\end{small}
\end{cpdflib}

\begin{pycpdflib}
\clearpage
\section*{Python Interface}
\begin{small}\tt
\lstinputlisting{pysplits/c07}
\end{small}
\end{pycpdflib}

\begin{dotnetcpdflib}
\clearpage
\section*{.NET Interface}
\begin{small}\tt
\lstinputlisting{dotnetsplits/c07}
\end{small}
\end{dotnetcpdflib}

\begin{jcpdflib}
\clearpage
\section*{Java Interface}
\begin{small}\tt
\lstinputlisting{javasplits/c07}
\end{small}
\end{jcpdflib}

\begin{jscpdflib}
\clearpage
\section*{JavaScript Interface}
\begin{small}\tt
\lstinputlisting{javascriptsplits/c07}
\end{small}
\end{jscpdflib}

\chapter{Presentations}\label{chap:7}\pagestyle{fancy}
  \begin{framed}
  \small\noindent\begin{verbatim}
  cpdf -presentation in.pdf [<range>] -o out.pdf
                     [-trans <transition-name>] [-duration <float>]
                     [-vertical] [-outward] [-direction <int>]
                     [-effect-duration <float>]\end{verbatim}
\end{framed}
\index{presentations}

  \vspace{12mm}
The PDF file format, starting at Version 1.1, provides for simple slide-show
presentations in the manner of Microsoft Powerpoint. These can be played in
Acrobat and possibly other PDF viewers, typically started by entering
full-screen mode. The \texttt{-presentation} operation allows such a
presentation to be built from any PDF file.

The \texttt{-trans} option chooses the transition style. When a page range is
used, it is the transition \textit{from} each page named which is altered. The
following transition styles are available:

\begin{description}
  \item[Split]Two lines sweep across the screen, revealing the new page. By
default the lines are horizontal. Vertical lines are selected by using the
\texttt{-vertical} option.
  \item[Blinds]Multiple lines sweep across the screen, revealing the new page.
By default the lines are horizontal. Vertical lines are selected by using the
\texttt{-vertical} option.
  \item[Box]A rectangular box sweeps inward from the edges of the page. Use
\texttt{-outward} to make it sweep from the center to the edges.
  \item[Wipe]A single line sweeps across the screen from one edge to the other
in a direction specified by the \texttt{-direction} option.
  \item[Dissolve]The old page dissolves gradually to reveal the new one.
  \item[Glitter]The same as \textbf{Dissolve} but the effect sweeps across the
page in the direction specified by the \texttt{-direction} option.
\end{description}

\noindent To remove a transition style currently applied to the selected pages,
omit the \texttt{-trans} option.

The \texttt{-effect-duration} option specifies the length of time in seconds
for the transition itself. The default value is one second.

The \texttt{-duration} option specifies the maximum time in seconds that the
page is displayed before the presentation automatically advances. The default,
in the absence of the \texttt{-duration} option, is for no automatic
advancement.

The \texttt{-direction} option (for \textbf{Wipe} and \textbf{Glitter} styles
only) specifies the direction of the effect. The following values are valid:
\begin{itemize}
  \item[\textbf{0}] Left to right
  \item[\textbf{90}] Bottom to top (\textbf{Wipe} only)
  \item[\textbf{180}] Right to left (\textbf{Wipe} only)
  \item[\textbf{270}] Top to bottom
  \item[\textbf{315}] Top-left to bottom-right (\textbf{Glitter} only)
\end{itemize}

\noindent For example:
\begin{framed}
  \small
  \noindent\verb!cpdf -presentation in.pdf 2-end -trans Split -duration 10 -o out.pdf!

  \vspace{2.5mm}
  The \textbf{Split} style, with vertical lines, and each slide staying ten
seconds unless manually advanced. The first page (being a title) does not move
on automatically, and has no transition effect.

\end{framed}

\noindent To use different options on different page ranges, run \cpdf\ multiple times on
the file using a different page range each time.

\begin{cpdflib}
\clearpage
\section*{C Interface}
\begin{small}\tt
\lstinputlisting{splits/c08}
\end{small}
\end{cpdflib}

\begin{pycpdflib}
\clearpage
\section*{Python Interface}
\begin{small}\tt
\lstinputlisting{pysplits/c08}
\end{small}
\end{pycpdflib}

\begin{dotnetcpdflib}
\clearpage
\section*{.NET Interface}
\begin{small}\tt
\lstinputlisting{dotnetsplits/c08}
\end{small}
\end{dotnetcpdflib}

\begin{jcpdflib}
\clearpage
\section*{Java Interface}
\begin{small}\tt
\lstinputlisting{javasplits/c08}
\end{small}
\end{jcpdflib}

\begin{jscpdflib}
\clearpage
\section*{JavaScript Interface}
\begin{small}\tt
\lstinputlisting{javascriptsplits/c08}
\end{small}
\end{jscpdflib}

\chapter{Watermarks and Stamps}\label{chap:8}
\label{stamps}
\index{watermarks}
  \begin{framed}
  \noindent\small\verb!cpdf -stamp-on source.pdf!\\
  \noindent\small\verb!     [-scale-stamp-to-fit] [<positioning command>] [-relative-to-cropbox] !\\
  \noindent\small\verb!     in.pdf [<range>] [-fast] -o out.pdf!
  
  \vspace{1.5mm}
  \noindent\small\verb!cpdf -stamp-under source.pdf!\\
  \noindent\small\verb!     [-scale-stamp-to-fit] [<positioning command>] [-relative-to-cropbox]!\\
  \noindent\small\verb!     in.pdf [<range>] [-fast] -o out.pdf!

  \vspace{1.5mm}
  \noindent\small\verb!cpdf -combine-pages over.pdf under.pdf!\\
  \noindent\small\verb!     [-fast] [-prerotate] [-no-warn-rotate] -o out.pdf!

  \vspace{1.5mm}
  \noindent\small\begin{verbatim}cpdf ([-add-text <text-format> | -add-rectangle <size>])
              [-font <fontname>]          [-font-size <size-in-points>]
              [-color <color>]            [-line-spacing <number>]
              [-outline]                  [-linewidth <number>]
              [-underneath]               [-relative-to-cropbox]
              [-prerotate]                [-no-warn-rotate]
              [-bates <number>]           [-bates-at-range <number>]
              [-bates-pad-to <number>]    [-opacity <number>]
              [-midline]                  [-topline]
              [-fast]
              in.pdf [<range>] -o out.pdf\end{verbatim}
  \noindent See also positioning commands below.

  \vspace{1.5mm}
  \noindent\small\verb!cpdf -remove-text in.pdf [<range>] -o out.pdf!

  \vspace{1.5mm}
  \noindent\small\verb!cpdf -prepend-content <content> in.pdf [<range>] -o out.pdf!

  \vspace{1.5mm}
  \noindent\small\verb!cpdf -postpend-content <content> in.pdf [<range>] -o out.pdf!

  \vspace{1.5mm}
  \noindent\small\verb!cpdf -stamp-as-xobject stamp.pdf in.pdf [<range>] -o out.pdf!

  \vspace{1.5mm}
  \noindent\small NB: See discussion of \texttt{-fast} in Section \ref{fast}.
  \end{framed}

  \section{Add a Watermark or Logo}
  The \texttt{-stamp-on} and \texttt{-stamp-under} operations stamp the first
page of a source PDF onto or under each page in the given range of the input
file. For example,
  \begin{framed}
    \small\verb!cpdf -stamp-on logo.pdf in.pdf odd -o out.pdf!
  \end{framed}
\noindent stamps the file \texttt{logo.pdf} onto the odd pages of \texttt{in.pdf},
writing to \texttt{out.pdf}. A watermark should go underneath each page:
  \begin{framed}
    \small\verb!cpdf -stamp-under topsecret.pdf in.pdf -o out.pdf!
  \end{framed}

\noindent The position commands in Section \ref{position} can be used to locate the stamp more precisely (they are calculated relative to the crop box of the stamp). Or, preprocess the stamp with \texttt{-shift} first.

The \texttt{-scale-stamp-to-fit} option can be added to scale the stamp to fit the page before applying it. The use of positioning commands together with \texttt{-scale-stamp-to-fit} is not recommended.

  The \texttt{-combine-pages} operation takes two PDF files and stamps each
page of one over each page of the other. The length of the output is the same
as the length of the ``under'' file. For instance:
  \begin{framed}
    \small\verb!cpdf -combine-pages over.pdf under.pdf -o out.pdf!
  \end{framed}

\noindent Page attributes (such as the display rotation) are taken from the ``under''
file. For best results, remove any rotation differences in the two files using
\texttt{-upright} (see above) first.

\noindent The \texttt{-relative-to-cropbox} option takes the positioning command to be relative to the crop box of each page rather than the media box.

  \section{Stamp Text, Dates and Times.}
\index{date}
\index{time}
\index{stamp text}
  The \texttt{-add-text} operation allows text, dates and times to be stamped
over one or more pages of the input at a given position and using a given font,
font size and color.
  \begin{framed}
    \small\verb!cpdf -add-text "Copyright 2014 ACME Corp." in.pdf -o out.pdf!
  \end{framed}
  \noindent The default is black 12pt Times New Roman text in the top left of each page. The text can be placed underneath rather than over the page by adding the \texttt{-underneath} option.
  
  Text previously added by \cpdf\ may be removed by the \texttt{-remove-text} operation.
\index{removing text}
  \begin{framed}
    \small\verb!cpdf -remove-text in.pdf -o out.pdf!
  \end{framed}

  \subsection{Page Numbers}
\index{page!numbers}
  There are various special codes to include the page number in the text:

  \vspace{2mm}
  \begin{tabular}{ll}
    \texttt{\%Page} & Page number in arabic notation (1, 2, 3\ldots) \\
    \texttt{\%PageDiv2} & Page number in arabic notation divided by two \\
    \texttt{\%roman} & Page number in lower-case roman notation (i, ii, iii\ldots) \\
    \texttt{\%Roman} & Page number in upper-case roman notation (I, II, III\ldots) \\
    \texttt{\%EndPage} & Last page of document in arabic notation \\
    \texttt{\%Label} & The page label of the page \\
    \texttt{\%EndLabel} & The page label of the last page \\
    \texttt{\%filename} & The full file name of the input document \\
  \end{tabular}

  \vspace{2mm}
  \noindent For example, the format \texttt{"Page~\%Page~of~\%EndPage"} might become "Page~5~of~17".

  NB: In some circumstances (e.g in batch files) on Microsoft Windows, \verb!%! is a special character, and must be escaped (written as \verb$%%$). Consult your local documentation for details.

  \subsection{Date and Time Formats}
  \begin{tabular}{ll}
    \texttt{\%a} & Abbreviated weekday name (Sun, Mon etc.)\\
    \texttt{\%A} & Full weekday name (Sunday, Monday etc.)\\
    \texttt{\%b} & Abbreviated month name (Jan, Feb etc.)\\
    \texttt{\%B} & Full month name (January, February etc.)\\
    \texttt{\%d} & Day of the month (01--31) \\
    \texttt{\%e} & Day of the month (1--31) \\
    \texttt{\%H} & Hour in 24-hour clock (00--23)\\
    \texttt{\%I} & Hour in 12-hour clock (01--12)\\
    \texttt{\%j} & Day of the year (001--366)\\
    \texttt{\%m} & Month of the year (01--12)\\
    \texttt{\%M} & Minute of the hour (00--59)\\
    \texttt{\%p} & "a.m" or "p.m"\\
    \texttt{\%S} & Second of the minute (00--61)\\
    \texttt{\%T} & Same as \%H:\%M:\%S\\
    \texttt{\%u} & Weekday (1--7, 1 = Sunday)\\
    \texttt{\%w} & Weekday (0--6, 0 = Sunday)\\
    \texttt{\%Y} & Year (0000--9999)\\
    \texttt{\%\%} & The \% character.
  \end{tabular}

  \subsection{Bates Numbers}
\index{bates numbers}
  Unique page identifiers can be specified by putting \verb!%Bates! in the format.
The starting point can be set with the \texttt{-bates} option. For example:
  \begin{framed}
    \small\verb!cpdf -add-text "Page ID: %Bates" -bates 23745 in.pdf -o out.pdf!
  \end{framed}

\noindent To specify that bates numbering begins at the first page of the range, use \texttt{-bates-at-range} instead. This option must be specified after the range is specified. To pad the bates number up to a given number of leading zeros, use \texttt{-bates-pad-to} in addition to either \texttt{-bates} or \texttt{-bates-at-range}.



  \subsection{Position}
  \label{position}
  The position of the text may be specified either in absolute terms:
  \begin{framed}
    \small\verb!-pos-center "200 200"!
  
    \vspace{2.5mm}
    \noindent Position the center of the baseline text at (200pt, 200pt)

    \vspace{2.5mm}
    \small\verb!-pos-left "200 200"!
  
    \vspace{2.5mm}
    \noindent Position the left of the baseline of the text at (200pt, 200pt)

    \vspace{2.5mm}
    \small\verb!-pos-right "200 200"!
  
    \vspace{2.5mm}
    \noindent Position the right of the baseline of the text at (200pt, 200pt)

  \end{framed}

  \noindent Positions relative to certain common points can be set:

  \begin{framed}
    \noindent\begin{tabular}{ll}
      \small\verb!-top 10! & Center of baseline 10 pts down from the top center \\
      \small\verb!-topleft 10! & Left of baseline 10 pts down and in from top left \\
      \small\verb!-topright 10! & Right of baseline 10 pts down and left from top right\\
      \small\verb!-left 10! & Left of baseline 10 pts in from center left \\
      \small\verb!-bottomleft 10! & Left of baseline 10 pts in and up from bottom left \\
      \small\verb!-bottom 10! & Center of baseline 10 pts up from bottom center\\
      \small\verb!-bottomright 10! & Right of baseline 10 pts up and in from bottom right \\
      \small\verb!-right 10! & Right of baseline 10 pts in from the center right \\
      \small\verb!-diagonal! & Diagonal, bottom left to top right, centered on page\\
      \small\verb!-reverse-diagonal! & Diagonal, top left to bottom right, centered on page\\    
      \small\verb!-center! & Centered on page\\
    \end{tabular}
  \end{framed}

\noindent No attempt is made to take account of the page rotation when interpreting the
position, so \texttt{-prerotate} may be added to the command line if the file
contains pages with a non-zero viewing rotation (to silence the rotation warning, add \texttt{-no-warn-rotate} instead) This is equivalent to
pre-processing the document with \texttt{-upright} (see chapter 3).
   
The \texttt{-relative-to-cropbox} modifier can be added to the command line to
make these measurements relative to the crop box instead of the media box.

The default position is equivalent to \texttt{-topleft 100}.

The \texttt{-midline} option may be added to specify that the positioning
commands above are to be considered relative to the midline of the text, rather
than its baseline. Similarly, the \texttt{-topline} option may be used to specify that the position is taken relative to the top of the text.

  \subsection{Font and Size}
\index{font}
  The font may be set with the \texttt{-font} option. The 14 Standard PDF fonts are available:

  \vspace{2mm}
  \begin{tabular}{l}
  Times-Roman\\
  Times-Bold\\
  Times-Italic\\
  Times-BoldItalic\\
  Helvetica\\
  Helvetica-Bold\\
  Helvetica-Oblique\\
  Helvetica-BoldOblique\\
  Courier\\
  Courier-Bold\\
  Courier-Oblique\\
  Courier-BoldOblique\\
  Symbol\\
  ZapfDingbats
  \end{tabular}
 \vspace{2mm}

  \noindent For example, page numbers in Times Italic can be achieved by:
  \begin{framed}
    \small\verb!cpdf -add-text "-%Page-" -font "Times-Italic" in.pdf -o out.pdf!
  \end{framed}
  \noindent See Section \ref{copyfont} for how to use other fonts. The font size can be altered with the \texttt{-font-size} option, which
specifies the size in points:
  \begin{framed}
    \small\verb!cpdf -add-text "-%Page-" -font-size 36 in.pdf -o out.pdf!
  \end{framed}

  \subsection{Colors}
\index{color}
  The \texttt{-color} option takes an RGB (3 values), CYMK (4 values), or Grey (1 value) color. Components range between 0 and 1. The following RGB colors are predefined:

  \vspace{2mm}
  \begin{tabular}{ll}
    \textbf{Color} & \textbf{R, G, B} \\ \hline
     white & 1, 1, 1\\
     black & 0, 0, 0\\
     red & 1, 0, 0\\
     green & 0, 1, 0\\
     blue & 0, 0, 1\\
  \end{tabular}

  \begin{framed}
    \small\verb!cpdf -add-text "Hullo" -color "red" in.pdf -o out.pdf!
    
    \vspace{1.5mm}
    \small\verb!cpdf -add-text "Hullo" -color "0.5 0.5 0.5" in.pdf -o out.pdf!

    \vspace{1.5mm}
    \small\verb!cpdf -add-text "Hullo" -color "0.75" in.pdf -o out.pdf!

    \vspace{1.5mm}
    \small\verb!cpdf -add-text "Hullo" -color "0.5 0.5 0.4 0.9" in.pdf -o out.pdf!
  \end{framed}

\noindent Partly-transparent text may be specified using the \verb!-opacity! option. Wholly opaque is 1 and wholly transparent is 0. For example:

\begin{framed}
  \small\verb!cpdf -add-text "DRAFT" -color "red" -opacity 0.3 -o out.pdf!
\end{framed}

\subsection{Outline Text}
\index{outline text}

  The \texttt{-outline} option sets outline text. The line width (default 1pt)
  may be set with the \texttt{-linewidth} option. For example, to stamp
  documents as drafts:

  \begin{framed}
    \small\verb!cpdf -add-text "DRAFT" -diagonal -outline in.pdf -o out.pdf!
    
  \end{framed}

\subsection{Multi-line Text}

The code \texttt{$\backslash$n} can be included in the text string to move to
the next line. In this case, the vertical position refers to the baseline of
the first line of text (if the position is at the top, top left or top right of
the page) or the baseline of the last line of text (if the position is at the
bottom, bottom left or bottom right).

  \begin{framed}
    \small\begin{verbatim}cpdf -add-text "Specification\n%Page of %EndPage"
               -topright 10 in.pdf -o out.pdf\end{verbatim}
  \end{framed}

\noindent The \texttt{-midline} option may be used to make these vertical positions
relative to the midline of a line of text rather than the baseline, as usual.

The \texttt{-line-spacing} option can be used to increase or decrease the line
spacing, where a spacing of 1 is the standard.

  \begin{framed}
    \small\begin{verbatim}cpdf -add-text "Specification\n%Page of %EndPage"
               -topright 10 -line-spacing 1.5 in.pdf -o out.pdf\end{verbatim}
  \end{framed}

\noindent Justification of multiple lines is handled by the \texttt{-justify-left}, 
\texttt{-justify-right} and\\ \texttt{-justify-center} options. The defaults are
left justification for positions relative to the left hand side of the page,
right justification for those relative to the right, and center justification
for positions relative to the center of the page. For example:

\begin{framed}
  \small\begin{verbatim}cpdf -add-text "Long line\nShort" -justify-right
               in.pdf -o out.pdf\end{verbatim}
\end{framed}

\subsection{Special Characters}

If your command line allows for the inclusion of unicode characters, the input
text will be considered as UTF8 by \verb!cpdf!. Special characters which exist
in the PDF WinAnsiEncoding Latin 1 code (such as many accented characters) will
be reproduced in the PDF. This does not mean, however, that every special
character can be reproduced -- it must exist in the font. When using a custom font, cpdf will attempt to convert from UTF8 to the encoding of that font automatically.

(For compatibility with previous versions of cpdf, special characters may be
introduced manually with a backslash followed by the three-digit octal code of
the character in the PDF WinAnsiEncoding Latin 1 Code. The full table is
included in Appendix D of the Adobe PDF Reference Manual, which is available at
\url{https://wwwimages2.adobe.com/content/dam/acom/en/devnet/pdf/pdfs/PDF32000_2008.pdf}. For example, a German sharp s (\ss) may be introduced by \verb!\337!. \textit{This functionality was withdrawn as of version 2.6})

\section{Stamping Graphics}

A rectangle may be placed on one or more pages by using the \texttt{-add-rectangle <size>} command. Most of the options discussed above for text placement apply in the same way. For example:

\begin{framed}
  \small\begin{verbatim}cpdf -add-rectangle "200 300" -pos-right 30 -color red -outline
                    in.pdf -o out.pdf\end{verbatim}
\end{framed}

\noindent This can be used to blank out or highlight part of the document. The following positioning options work as you would expect: \texttt{-topleft}, \texttt{-top}, \texttt{-topright}, \texttt{-right}, \texttt{-bottomright}, \texttt{-bottom}, \texttt{-bottomleft}, \texttt{-left}, \texttt{-center}. When using the option \texttt{-pos-left "x y"}, the point (x, y) refers to the bottom-left of the rectangle. When using the option \texttt{-pos-right "x y"}, the point (x, y) refers to the bottom-right of the rectangle. When using the option \texttt{-pos-center "x y"}, the point (x, y) refers to the center of the rectangle. The options \texttt{-diagonal} and \texttt{-reverse-diagonal} have no meaning.\pagestyle{empty}\thispagestyle{fancy}

\section{Low-level facilities}

These two operations add content directly to the beginning or end of the page data for a page. You must understand the PDF page description language to use these.

\begin{framed}
  \noindent\small\verb!cpdf -prepend-content <content> in.pdf [<range>] -o out.pdf!\\

\vspace{1.5mm}
  \noindent\small\verb!cpdf -postpend-content <content> in.pdf [<range>] -o out.pdf!
\end{framed}

\noindent The \texttt{-fast} option may be added (see Chapter 1). The \texttt{-stamp-as-xobject} operation puts a file in another as a Form XObject on the given pages. You can then use \texttt{-prepend-content} or \texttt{-postpend-content} to use it.

\begin{framed}
  \noindent\small\verb!cpdf -stamp-as-xobject stamp.pdf in.pdf [<range>] -o out.pdf!
\end{framed}

\begin{cpdflib}
\clearpage
\section*{C Interface}
\begin{small}\tt
\lstinputlisting{splits/c09}
\end{small}
\end{cpdflib}

\begin{pycpdflib}
\clearpage
\section*{Python Interface}
\begin{small}\tt
\lstinputlisting{pysplits/c09}
\end{small}
\end{pycpdflib}

\begin{dotnetcpdflib}
\clearpage
\section*{.NET Interface}
\begin{small}\tt
\lstinputlisting{dotnetsplits/c09}
\end{small}
\end{dotnetcpdflib}

\begin{jcpdflib}
\clearpage
\section*{Java Interface}
\begin{small}\tt
\lstinputlisting{javasplits/c09}
\end{small}
\end{jcpdflib}

\begin{jscpdflib}
\clearpage
\section*{JavaScript Interface}
\begin{small}\tt
\lstinputlisting{javascriptsplits/c09}
\end{small}
\end{jscpdflib}

\chapter{Multipage Facilities}\pagestyle{fancy}\label{multipage}\label{chap:9}
  \begin{framed}
    \small\noindent\verb!cpdf -pad-before in.pdf [<range>] [-pad-with pad.pdf] -o out.pdf!

    \vspace{1.5mm}
    \small\noindent\verb!cpdf -pad-after in.pdf [<range>] [-pad-with pad.pdf] -o out.pdf!

    \vspace{1.5mm}
    \small\noindent\verb!cpdf -pad-every [<integer>] in.pdf [-pad-with pad.pdf] -o out.pdf!

    \vspace{1.5mm}
    \small\noindent\verb!cpdf -pad-multiple [<integer>] in.pdf -o out.pdf!

    \vspace{1.5mm}
    \small\noindent\verb!cpdf -pad-multiple-before [<integer>] in.pdf -o out.pdf!

    \vspace{1.5mm}
    \small\noindent\verb!cpdf [-impose <pagesize> | impose-xy <x y>]!\\
    \small\noindent\verb!     [-impose-columns] [-impose-rtl] [-impose-btt]!\\
    \small\noindent\verb!     [-impose-margin <margin>] [-impose-spacing <spacing>]!\\
    \small\noindent\verb!     [-impose-linewidth <width>] [-fast]!\\
    \small\noindent\verb!     in.pdf -o out.pdf!

    \vspace{1.5mm}
    \small\noindent\verb!cpdf -twoup-stack [-fast] in.pdf -o out.pdf! 

    \vspace{1.5mm}
    \small\noindent\verb!cpdf -twoup [-fast] in.pdf -o out.pdf! 

  \end{framed}

  \section{Inserting Blank Pages}
\index{blank pages!inserting}
  Sometimes, for instance to get a printing arrangement right, it's useful to
be able to insert blank pages into a PDF file. \cpdf\ can add blank pages
before a given page or pages, or after. The pages in question are specified by
a range in the usual way:
  \begin{framed}
    \small\verb!cpdf -pad-before in.pdf 1 -o out.pdf!
 
    \vspace{2.5mm}
    \noindent Add a blank page before page 1 (i.e. at the beginning of the document.)

    \vspace{2.5mm}
    \verb!cpdf -pad-after in.pdf 2,16,38,84,121,147 -o out.pdf!

    \vspace{2.5mm}
    \noindent Add a blank page after pages 2, 16, 38, 84, 121 and 147 (for
instance, to add a clean page between chapters of a document.)
  \end{framed}
  \noindent The dimensions of the padded page are derived from the boxes (media box, crop box etc.) of the page after or before which the padding is to be applied.

  The \verb!-pad-every n! operation places a blank page after every n pages, excluding any last one. For example\ldots
  \begin{framed}
    \small\verb!cpdf -pad-every 3 in.pdf -o out.pdf!
 
    \vspace{2.5mm}
    \noindent Add a blank page after every three pages
  \end{framed}
  \noindent\ldots on a 9 page document adds a blank page after pages 3 and 6.

In all three of these operations, one may specify \texttt{-pad-with} providing a (usually one-page)  PDF file to be used instead of a blank page. For example, a page saying ``This page left intentionally blank''.

  The \verb!-pad-multiple n! operation adds blank pages so the document has a multiple of \verb!n! pages. For example:
  \begin{framed}
    \small\verb!cpdf -pad-multiple 8 in.pdf -o out.pdf!
 
    \vspace{2.5mm}
    \noindent Add blank pages to \texttt{in.pdf} so it has a multiple of 8 pages. 
  \end{framed}
 

\noindent The \texttt{-pad-multiple-before n} operation adds the padding pages at the beginning of the file  instead.

\section{Imposition}

\index{two-up}\index{imposition}

Imposition is the act of putting two or more pages of an input document onto each page of the output document. There are two operations provided by \texttt{cpdf}:

\begin{itemize}
\item the \texttt{-impose} operation which, given a page size fits multiple pages into it; and
\item the \texttt{-impose-xy} operation which, given an $x$ and $y$ value, builds an output page which fits $x$ input pages horizontally and $y$ input pages vertically. \end{itemize}

  \begin{framed}
    \small\verb!cpdf -impose a0landscape in.pdf -o out.pdf!
 
    \vspace{2.5mm}
    \noindent Impose as many pages as will fit on to new A0 landscape pages. 
  \end{framed}

  \begin{framed}
    \small\verb!cpdf -impose-xy "3 4" in.pdf -o out.pdf!
 
    \vspace{2.5mm}
    \noindent Impose 3 across and 4 down on to new pages of 3 times the width and 4 times the height of the input ones. 
  \end{framed}

The $x$ value for \texttt{-impose-xy} may be set to zero to indicate an infinitely-wide page; the $y$ value to indicate an infinitely-long one.

In both cases, the pages in the input file are assumed to be of the same dimensions.

The following options may be used to modify the output:

\begin{itemize}
\item \texttt{-impose-columns} Lay the pages out in columns rather than rows.
\item \texttt{-impose-rtl} Lay the pages out right-to-left.
\item \texttt{-impose-btt} Lay the pages out bottom-to-top.
\item \texttt{-impose-margin <margin>} Add a margin around the edge of the page of the given width. When using \texttt{-impose-xy} the page size increases; with \texttt{-impose} the pages are scaled.
\item \texttt{-impose-spacing <spacing>} Add spacing between each row and column. When using \texttt{-impose-xy} the page size increases; with \texttt{-impose} the pages are scaled.
\item \texttt{-impose-linewidth <width>} Add a border around each input page. With \texttt{-impose} the pages are scaled after the border is added, so you must account for this yourself.

\end{itemize}

To impose with rotated pages, for example to put two A4 portrait pages two-up on an A3 landscape page, rotate them prior to imposition.

Two other ways of putting multiple pages on a single page remain from earlier versions of \texttt{cpdf} which lacked a general imposition operation.  The \texttt{-twoup-stack} operation puts two logical pages on each physical
page, rotating them 90 degrees to do so. The new mediabox is thus larger. The \texttt{-twoup} operation does the same, but scales the new sides down so
that the media box is unchanged.

  \begin{framed}
    \small\verb!cpdf -twoup in.pdf -o out.pdf!
 
    \vspace{2.5mm}
    \noindent Impose a document two-up, keeping the existing page size.

    \small\verb!cpdf -twoup-stack in.pdf -o out.pdf!
 
    \vspace{2.5mm}
    \noindent Impose a document two-up on a larger page by rotation. 
  \end{framed}
 
NB: For all imposition options, see also discussion of \texttt{-fast} in Section \ref{fast}.


\begin{cpdflib}
\clearpage
\section*{C Interface}
\begin{small}\tt
\lstinputlisting{splits/c10}
\end{small}
\end{cpdflib}

\begin{pycpdflib}
\clearpage
\section*{Python Interface}
\begin{small}\tt
\lstinputlisting{pysplits/c10}
\end{small}
\end{pycpdflib}

\begin{dotnetcpdflib}
\clearpage
\section*{.NET Interface}
\begin{small}\tt
\lstinputlisting{dotnetsplits/c10}
\end{small}
\end{dotnetcpdflib}

\begin{jcpdflib}
\clearpage
\section*{Java Interface}
\begin{small}\tt
\lstinputlisting{javasplits/c10}
\end{small}
\end{jcpdflib}

\begin{jscpdflib}
\clearpage
\section*{JavaScript Interface}
\begin{small}\tt
\lstinputlisting{javascriptsplits/c10}
\end{small}
\end{jscpdflib}

\chapter{Annotations}\label{chap:10}
  \begin{framed}
  \small\noindent\verb!cpdf -list-annotations in.pdf [<range>]!

  \vspace{1.5mm}
  \small\noindent\verb!cpdf -list-annotations-json in.pdf [<range>]!

  \vspace{1.5mm}
  \small\noindent\verb!cpdf -copy-annotations from.pdf to.pdf [<range>] -o out.pdf!

  \vspace{1.5mm}
  \small\noindent\verb!cpdf -remove-annotations in.pdf [<range>] -o out.pdf!
  \end{framed}

  \section{Listing Annotations}
\index{annotations!listing}\index{JSON!list annotations as}
  The \texttt{-list-annotations} operation prints the textual content of any
annotations on the selected pages to standard output. Each annotation is preceded by the page number and followed by a newline. The output of this operation is always UTF8.
  \begin{framed}
    \small\verb!cpdf -list-annotations in.pdf > annots.txt!
    
    \vspace{2.5mm}
    \noindent Print annotations from \texttt{in.pdf}, redirecting output to \texttt{annots.txt}.
  \end{framed}

More information can be obtained by listing annotations in JSON format:

  \begin{framed}
    \small\verb!cpdf -list-annotations-json in.pdf > annots.json!
    
    \vspace{2.5mm}
    \noindent Print annotations from \texttt{in.pdf} in JSON format, redirecting output to \texttt{annots.json}.
  \end{framed}

This produces an array of (page number, annotation) pairs giving the PDF structure of each annotation. Destination pages for page links will have page numbers in place of internal PDF page links, and certain indirect objects are made direct but the content is otherwise unaltered. Here is an example entry for an annotation on page 10:

{\small\begin{verbatim}
  [
  10,
  { "/H": { "N": "/I" },
    "/Border": [ { "I": 0 }, { "I": 0 }, { "I": 0 } ],
    "/Rect": [
      { "F": 89.88023 }, { "F": 409.98401 }, { "F": 323.90561 }, {
        "F": 423.32059 } ],
    "/Subtype": { "N": "/Link" },
    "/Type": { "N": "/Annot" },
    "/A": {
      "/S": { "N": "/URI" },
      "/URI": "http://www.google.com/" },
    "/StructParent": { "I": 10 } } ]
\end{verbatim}}

A future version of \texttt{cpdf} will allow these JSON annotations to be edited and re-loaded into a PDF file.

  \section{Copying Annotations}
\index{annotations!copying}
  The \texttt{-copy-annotations} operation copies the annotations in the given
page range from one file (the file specified immediately after the option) to
another pre-existing PDF. The range is specified after this pre-existing PDF.
The result is then written an output file, specified in the usual way.
  \begin{framed}
    \small\verb!cpdf -copy-annotations from.pdf to.pdf 1-10 -o result.pdf !
    
    \vspace{2.5mm}
    \noindent Copy annotations from the first ten pages of \texttt{from.pdf}
onto the PDF file \texttt{to.pdf}, writing the result to \texttt{results.pdf}.

  \end{framed}

  \section{Removing Annotations}
\index{annotations!removing}
  The \texttt{-remove-annotations} operation removes all annotations from the
given page range.

  \begin{framed}
    \small\verb!cpdf -remove-annotations in.pdf 1 -o out.pdf!
    
    \vspace{2.5mm}
    \noindent Remove annotations from the first page of a file only.
  \end{framed}

\begin{cpdflib}
\clearpage
\section*{C Interface}
\begin{small}\tt
\lstinputlisting{splits/c11}
\end{small}
\end{cpdflib}

\begin{pycpdflib}
\clearpage
\section*{Python Interface}
\begin{small}\tt
\lstinputlisting{pysplits/c11}
\end{small}
\end{pycpdflib}

\begin{dotnetcpdflib}
\clearpage
\section*{.NET Interface}
\begin{small}\tt
\lstinputlisting{dotnetsplits/c11}
\end{small}
\end{dotnetcpdflib}

\begin{jcpdflib}
\clearpage
\section*{Java Interface}
\begin{small}\tt
\lstinputlisting{javasplits/c11}
\end{small}
\end{jcpdflib}

\begin{jscpdflib}
\clearpage
\section*{JavaScript Interface}
\begin{small}\tt
\lstinputlisting{javascriptsplits/c11}
\end{small}
\end{jscpdflib}

\chapter{Document Information and Metadata}\label{chap:11}
\index{document information}
\index{metadata}
  \begin{framed}

    \small\noindent\verb!cpdf -info [-raw | -utf8] in.pdf!

    \vspace{1.5mm}
    \small\noindent\verb!cpdf -page-info in.pdf!

    \vspace{1.5mm}
    \small\noindent\verb!cpdf -pages in.pdf!

    \vspace{1.5mm}
    \small\noindent\verb!cpdf -set-title <title of document>!\\
    \small\noindent\verb!     [-also-set-xmp] [-just-set-xmp] [-raw] in.pdf -o out.pdf!\\
    (Also \texttt{-set-author} etc. See Section \ref{setdocinfo}.)

    \vspace{1.5mm}
    \small\noindent\verb!cpdf -set-page-layout <layout> in.pdf -o out.pdf!

    \vspace{1.5mm}
    \small\noindent\verb!cpdf -set-page-mode <mode> in.pdf -o out.pdf!

    \vspace{1.5mm} 
    \small\noindent\verb!cpdf -hide-toolbar <true | false> in.pdf -o out.pdf!\\
    \noindent\verb!     -hide-menubar!\\
    \noindent\verb!     -hide-window-ui!\\
    \noindent\verb!     -fit-window!\\
    \noindent\verb!     -center-window!\\
    \noindent\verb!     -display-doc-title!
    
    \vspace{1.5mm}
    \small\noindent\verb!cpdf -open-at-page <page number> in.pdf -o out.pdf!\\
    \noindent\verb!cpdf -open-at-page-fit <page number> in.pdf -o out.pdf!

    \vspace{1.5mm}
    \small\noindent\verb!cpdf -set-metadata <metadata-file> in.pdf -o out.pdf!
    \small\noindent\verb!cpdf -remove-metadata in.pdf -o out.pdf!\\
    \small\noindent\verb!cpdf -print-metadata in.pdf!\\
    \small\noindent\verb!cpdf -create-metadata in.pdf -o out.pdf!\\
    \small\noindent\verb!cpdf -set-metadata-date <date> in.pdf -o out.pdf!
    
    \vspace{1.5mm}
    \small\noindent\verb!cpdf -add-page-labels in.pdf -o out.pdf!\\
    \noindent\verb!     [-label-style <style>] [-label-prefix <string>]!\\
    \noindent\verb!     [-label-startval <integer>] [-labels-progress]!\\
    
    \vspace{1.5mm}
    \small\noindent\verb!cpdf -remove-page-labels in.pdf -o out.pdf!\\
    \small\noindent\verb!cpdf -print-page-labels in.pdf!
  \end{framed}
 
\section{Reading Document Information}
\label{info}
The \texttt{-info} operation prints entries from the document information
dictionary, and from any XMP metadata to standard output.

\begin{framed}
{\small\begin{verbatim}
$cpdf -info pdf_reference.pdf
Encryption: 40bit
Linearized: true
Permissions: No edit
Version: 1.6
Pages: 1310
Title: PDF Reference, version 1.7
Author: Adobe Systems Incorporated
Subject: Adobe Portable Document Format (PDF)
Keywords: 
Creator: FrameMaker 7.2
Producer: Acrobat Distiller 7.0.5 (Windows)
Created: D:20061017081020Z
Modified: D:20061118211043-02'30'
XMP pdf:Producer: Adobe PDF library 7.77
XMP xmp:CreateDate: 2006-12-21T18:19:09+01:00
XMP xmp:CreatorTool: Adobe Illustrator CS2
XMP xmp:MetadataDate: 2006-12-21T18:19:09Z
XMP xmp:ModifyDate: 2006-12-21T18:19:09Z
XMP dc:title: AI6\end{verbatim}}\end{framed}
\noindent The details of the format for creation and modification dates can be found in
Appendix~\ref{dates}.

By default, cpdf strips to ASCII, discarding character codes in excess of 127. In order to preserve the original unicode, add the \texttt{-utf8} option. To disable all postprocessing of the string, add \texttt{-raw}. See Section \ref{textencodings} for more information.

\vspace{4mm}
The \texttt{-page-info} operation prints the page label, media box and other boxes
page-by-page to standard output, for all pages in the current range.

\begin{framed}
{\small\begin{verbatim}
$cpdf -page-info 14psfonts.pdf
Page 1:
Label: i
MediaBox: 0.000000 0.000000 600.000000 450.000000
CropBox: 200.000000 200.000000 500.000000 500.000000
BleedBox: 
TrimBox: 
ArtBox:
Rotation: 0
\end{verbatim}}
\end{framed}

\noindent Note that the format for boxes is minimum x, minimum y, maximum x, maximum y.

\smallgap 
\noindent The \texttt{-pages} operation prints the number of pages in the file.
\begin{framed}
{\small\begin{verbatim}
cpdf -pages Archos.pdf
8
\end{verbatim}}
\end{framed}

\section{Setting Document Information}
\label{setdocinfo}
  The \textit{document information dictionary} in a PDF file specifies various
pieces of information about a PDF. These can be consulted in a PDF viewer (for
instance, Acrobat).

  Here is a summary of the commands for setting entries in the document
information dictionary:

{\small\begin{framed}
    \noindent\begin{tabular}{ll}
       \textbf{Information} & \textbf{Example command-line fragment} \\
       Title & \texttt{cpdf -set-title "Discourses"} \\
       Author & \texttt{cpdf -set-author "Joe Smith"} \\
       Subject & \texttt{cpdf -set-subject "Behavior"} \\
       Keywords & \texttt{cpdf -set-keywords "Ape Primate"} \\
       Creator & \texttt{cpdf -set-creator "Original Program"} \\
       Producer & \texttt{cpdf -set-producer "Distilling Program"} \\
       Creation Date & \texttt{cpdf -set-create "D:19970915110347-08'00'"} \\
       Modification Date & \texttt{cpdf -set-modify "D:19970915110347-08'00'"} \\
       Mark as Trapped & \texttt{cpdf -set-trapped} \\
       Mark as Untrapped & \texttt{cpdf -set-untrapped} \\
    \end{tabular}
\end{framed}}

  \noindent (The details of the format for creation and modification dates can be found
in Appendix~\ref{dates}. Using the date \texttt{"now"} uses the time and date
at which the command is executed. Note also that \texttt{-producer} and \texttt{-creator} may be used to set the producer and/or the creator when writing any file, separate from the operations described in this chapter.)
  
  \vspace{2mm}
  For example, to set the title, the full command line would be
  \begin{framed}
    \small\verb!cpdf -set-title "A Night in London" in.pdf -o out.pdf!
  \end{framed}
\noindent The text string is considered to be in UTF8 format, unless the \texttt{-raw}
option is added---in which case, it is unprocessed, save for the replacement of any octal escape sequence such as \texttt{\textbackslash 017}, which is replaced by a character of its value (here, 15).

To set also any field in the XMP metadata, add \texttt{-also-set-xmp}. The field must exist already. To set only the field (not the document information dictionary), add \texttt{-just-set-xmp} instead.

To delete existing non-XMP metadata in line with PDF 2.0, use \texttt{-remove-dict-entry "/Info"} as described in chapter \ref{chap:misc}.

  \section{XMP Metadata}
\index{metadata!XMP}\index{XMP metadata}
  PDF files can contain a piece of arbitrary metadata, often in XMP format.
This is typically stored in an uncompressed stream, so that other applications
can read it without having to decode the whole PDF. To set the metadata:
  \begin{framed}
    \small\verb!cpdf -set-metadata data.xml in.pdf -o out.pdf!
  \end{framed}
  \noindent To remove any metadata:
  \begin{framed}
    \small\verb!cpdf -remove-metadata in.pdf -o out.pdf!
  \end{framed}
  \noindent To print the current metadata to standard output:
  \begin{framed}
    \small\verb!cpdf -print-metadata in.pdf!
  \end{framed}
  \noindent To create XMP metadata from scratch, using any information in the Document Information Dictionary (old-style metadata):
  \begin{framed}
    \small\verb!cpdf -create-metadata in.pdf -o out.pdf!
  \end{framed}
  \noindent To set the XMP metadata date field, use:
  \begin{framed}
    \small\verb!cpdf -set-metadata-date <date> in.pdf -o out.pdf!
  \end{framed}
\noindent The date format is defined in Appendix \ref{xmpdate}. Using the date \texttt{"now"} uses the time and date
at which the command is executed.


\section{Upon Opening a Document}

  \subsection{Page Layout}
\index{page!layout}
  The \texttt{-set-page-layout} operation specifies the page layout to be used
when a document is opened in, for instance, Acrobat. The possible
(case-sensitive) values are:

\vspace{2mm}
  {\small\begin{tabular}{ll}
    \texttt{SinglePage} & \vspace{2mm} \parbox{8cm}{Display one page at a time} \\
    \texttt{OneColumn} & \vspace{2mm} \parbox{8cm}{Display the pages in one column} \\
    \texttt{TwoColumnLeft} & \vspace{2mm} \parbox{8cm}{Display the pages in two columns, odd numbered pages on the left} \\
    \texttt{TwoColumnRight} & \vspace{2mm} \parbox{8cm}{Display the pages in two columns, even numbered pages on the left} \\
    \texttt{TwoPageLeft} & \vspace{2mm} \parbox{8cm}{(PDF 1.5 and above) Display the pages two at a time, odd numbered pages on the left} \\
    \texttt{TwoPageRight} & \vspace{2mm} \parbox{8cm}{(PDF 1.5 and above) Display the pages two at a time, even numbered pages on the left}
  \end{tabular}}\\

  \noindent For instance:
  \begin{framed}
    \small\verb!cpdf -set-page-layout TwoColumnRight in.pdf -o out.pdf!
  \end{framed}
  
\noindent NB: If the file has a valid \texttt{/OpenAction} setting, which tells the PDF reader to open at a certain page or position on a page, this will override the page layout option. To prevent this, use the \texttt{-remove-dict-entry} functionality from Section \ref{removedictentry}:

  \begin{framed}
    \small\verb!cpdf -remove-dict-entry /OpenAction in.pdf -o out.pdf!
  \end{framed}

  \subsection{Page Mode}
\index{page!mode}
  The \textit{page mode} in a PDF file defines how a viewer should display the
document when first opened. The possible (case-sensitive) values are:

\vspace{2mm}
  {\small\begin{tabular}{ll}
    \texttt{UseNone} & \vspace{2mm} \parbox{8cm}{Neither document outline nor thumbnail images visible} \\
    \texttt{UseOutlines} & \vspace{2mm} \parbox{8cm}{Document outline (bookmarks) visible} \\
    \texttt{UseThumbs} & \vspace{2mm} \parbox{8cm}{Thumbnail images visible} \\
    \texttt{FullScreen} & \vspace{2mm} \parbox{8cm}{Full-screen mode (no menu bar, window controls, or anything but the document visible)} \\
    \texttt{UseOC} & \vspace{2mm} \parbox{8cm}{(PDF 1.5 and above) Optional content group panel visible} \\
    \texttt{UseAttachments} & \vspace{2mm} \parbox{8cm}{(PDF 1.5 and above) Attachments panel visible}
  \end{tabular}}\\

  \noindent For instance:
  \begin{framed}
    \small\verb!cpdf -set-page-mode FullScreen in.pdf -o out.pdf!
  \end{framed}
  
  \subsection{Display Options}
\vspace{2mm}
  {\small\begin{tabular}{ll}
    \texttt{-hide-toolbar} & \vspace{2mm} \parbox{8cm}{Hide the viewer's toolbar} \\
    \texttt{-hide-menubar} & \vspace{2mm} \parbox{8cm}{Document outline (bookmarks) visible} \\
    \texttt{-hide-window-ui} & \vspace{2mm} \parbox{8cm}{Hide the viewer's scroll bars} \\
    \texttt{-fit-window} & \vspace{2mm} \parbox{8cm}{Resize the document's windows to fit size of first page} \\
    \texttt{-center-window} & \vspace{2mm} \parbox{8cm}{Position the document window in the center of the screen} \\
    \texttt{-display-doc-title} & \vspace{2mm} \parbox{8cm}{Display the document title instead of the file name in the title bar}
  \end{tabular}}\\

  \noindent For instance:
  \begin{framed}
    \small\verb!cpdf -hide-toolbar true in.pdf -o out.pdf!
  \end{framed}

\noindent The page a PDF file opens at can be set using \texttt{-open-at-page}:
  \begin{framed}
    \small\verb!cpdf -open-at-page 15 in.pdf -o out.pdf!
  \end{framed}

\noindent To have that page scaled to fit the window in the viewer, use \texttt{-open-at-page-fit} instead:
  \begin{framed}
    \small\verb!cpdf -open-at-page-fit end in.pdf -o out.pdf!
  \end{framed}

\noindent (Here, we used \texttt{end} to open at the last page. Any page specification describing a single page is ok here.)

\section{Page Labels}
\index{page!labels}

It is possible to add \textit{page labels} to a document. These are not the printed on the page, but may be displayed alongside thumbnails or in print dialogue boxes by PDF readers. We use \texttt{-add-page-labels} to do this, by default with decimal arabic numbers (1,2,3\ldots). We can add \texttt{-label-style} to choose what type of labels to add from these kinds:

\vspace{4mm}
{\small\begin{tabular}{rl}
  \texttt{DecimalArabic} & 1, 2, 3, 4, 5\ldots \\
  \texttt{LowercaseRoman} & i, ii, iii, iv, v\ldots \\
  \texttt{UppercaseRoman} & I, II, III, IV, V\ldots \\
  \texttt{LowercaseLetters} & a, b, c, \ldots , z, aa, bb\ldots \\
  \texttt{UppercaseLetters} & A, B, C, \ldots , Z, AA, BB\ldots \\
  \texttt{NoLabelPrefixOnly} & No number, but a prefix will be used if defined.
\end{tabular}}
\vspace{4mm}

\noindent We can use \texttt{-label-prefix} to add a textual prefix to each label. 
Consider a file with twenty pages and no current page labels (a PDF reader will assume 1,2,3\ldots if there are none). We will add the following page labels:

\vspace{4mm}
i, ii, iii, iv, 1, 2, 3, 4, 5, 6, 7, 8, 9, 10, A-0, A-1, A-2, A-3, A-4, A-5
\vspace{4mm}

\noindent Here are the commands, in order:

{\small\begin{framed}
  \noindent\verb!cpdf -add-page-labels in.pdf 1-4 -label-style LowercaseRoman!\\
  \noindent\verb!     -o out.pdf!\\
  
  \noindent\verb!cpdf -add-page-labels out.pdf 5-14 -o out.pdf!\\

  \noindent\verb!cpdf -add-page-labels out.pdf 15-20 -label-prefix "A-"!\\
  \noindent\verb!     -label-startval 0 -o out.pdf!
\end{framed}}

\noindent By default the labels begin at page number 1 for each range. To override this, we can use \texttt{-label-startval} (we used $0$ in the final command), where we want the numbers to begin at zero rather than one. The option \texttt{-labels-progress} can be added to make sure the  start value progresses between sub-ranges when the page range specified is disjoint, e.g \texttt{1-9, 30-40} or \texttt{odd}.

Page labels may be removed altogether by using \texttt{-remove-page-labels} command. To print the page labels from an existing file, use \texttt{-print-page-labels}. For example:
\begin{framed}\small\begin{verbatim}$ cpdf -print-page-labels cpdfmanual.pdf
labelstyle: LowercaseRoman
labelprefix: None
startpage: 1
startvalue: 1
labelstyle: DecimalArabic
labelprefix: None
startpage: 9
startvalue: 1
\end{verbatim}
\end{framed}\pagestyle{empty}\thispagestyle{fancy}

\begin{cpdflib}
\clearpage
\section*{C Interface}
\begin{small}\tt
\lstinputlisting{splits/c12}
\end{small}
\end{cpdflib}

\begin{pycpdflib}
\clearpage
\section*{Python Interface}
\begin{small}\tt
\lstinputlisting{pysplits/c12}
\end{small}
\end{pycpdflib}

\begin{dotnetcpdflib}
\clearpage
\section*{.NET Interface}
\begin{small}\tt
\lstinputlisting{dotnetsplits/c12}
\end{small}
\end{dotnetcpdflib}

\begin{jcpdflib}
\clearpage
\section*{Java Interface}
\begin{small}\tt
\lstinputlisting{javasplits/c12}
\end{small}
\end{jcpdflib}

\begin{jscpdflib}
\clearpage
\section*{JavaScript Interface}
\begin{small}\tt
\lstinputlisting{javascriptsplits/c12}
\end{small}
\end{jscpdflib}

\chapter{File Attachments}\label{chap:12}\pagestyle{fancy}
\index{attachments}
\begin{framed}
  \small\noindent\verb!cpdf -attach-file <filename> [-to-page <page number>] in.pdf -o out.pdf!

  \vspace{1.5mm}
  \small\noindent\verb!cpdf -list-attached-files in.pdf!
 
  \vspace{1.5mm}
  \small\noindent\verb!cpdf -remove-files in.pdf -o out.pdf!

  \vspace{1.5mm}
  \small\noindent\verb!cpdf -dump-attachments in.pdf -o <directory>!
\end{framed}
  PDF supports adding attachments (files of any kind, including other PDFs) to
an existing file. The \cpdf\ tool supports adding and removing \textit{document-level
attachments} --- that is, ones which are associated with the document as a
whole rather than with an individual page, and also \textit{page-level attachments}, associated with a particular page.
  \section{Adding Attachments}
\index{attachments!adding}
  To add an attachment, use the \texttt{-attach-file} operation. For instance,
  \begin{framed}
  \small\verb!cpdf -attach-file sheet.xls in.pdf -o out.pdf!
  \end{framed}
  \noindent attaches the Excel spreadsheet \texttt{sheet.xls} to the input file. If the file already has attachments, the new file is added to their number. You can specify multiple files to be attached by using \verb!-attach-file! multiple times. They will be attached in the given order.
  
  The \texttt{-to-page} option can be used to specify that the files will be attached to the given page, rather than at the document level. The \texttt{-to-page} option may be specified at most once. 

\section{Listing Attachments}
\index{attachments!listing}
To list all document- and page-level attachments, use the \texttt{-list-attached-files} operation. The page number and filename of each attachment is given, page 0 representing a document-level attachment.
\begin{framed}
{\small\begin{verbatim}
$cpdf -list-attached-files 14psfonts.pdf
0 utility.ml
0 utility.mli
4 notes.xls
\end{verbatim}}
\end{framed}

  \section{Removing Attachments}
\index{attachments!removing}
   To remove all document-level and page-level attachments from a file, use the \texttt{-remove-files} operation:
  \begin{framed}
    \small\verb!cpdf -remove-files in.pdf -o out.pdf!
  \end{framed}

\section{Dumping Attachments to File}
\index{attachments!dumping to file}

The \texttt{-dump-attachments} operation, when given a PDF file and a directory path as the output, will write each attachment under its filename (as displayed by \texttt{-list-attached-files} to that directory. The directory must exist prior to the call.

  \begin{framed}
    \small\verb!cpdf -dump-attachments in.pdf -o /home/fred/attachments!
  \end{framed}

\noindent  Unless the \texttt{-raw} option is given, the filenames are stripped of dubious special characters before writing. It is converted from unicode to 7 bit ASCII, and the following characters are removed, in addition to any character with ASCII code less than 32:
  \begin{framed}
  \centering
  \verb! / ? < > \ : * | " ^ + =!
  \end{framed}

\begin{cpdflib}
\clearpage
\section*{C Interface}
\begin{small}\tt
\lstinputlisting{splits/c13}
\end{small}
\end{cpdflib}

\begin{pycpdflib}
\clearpage
\section*{Python Interface}
\begin{small}\tt
\lstinputlisting{pysplits/c13}
\end{small}
\end{pycpdflib}

\begin{dotnetcpdflib}
\clearpage
\section*{.NET Interface}
\begin{small}\tt
\lstinputlisting{dotnetsplits/c13}
\end{small}
\end{dotnetcpdflib}

\begin{jcpdflib}
\clearpage
\section*{Java Interface}
\begin{small}\tt
\lstinputlisting{javasplits/c13}
\end{small}
\end{jcpdflib}

\begin{jscpdflib}
\clearpage
\section*{JavaScript Interface}
\begin{small}\tt
\lstinputlisting{javascriptsplits/c13}
\end{small}
\end{jscpdflib}

\chapter{Working with Images}\label{chap:13}
\begin{framed}
\noindent\small\verb!cpdf -extract-images in.pdf [<range>] [-im <path>] [-p2p <path>]!
\noindent\small\verb!     [-dedup | -dedup-perpage] -o <path>!

\vspace{1.5mm}
\noindent\small\verb!cpdf -image-resolution <minimum resolution> in.pdf [<range>]!
\end{framed}

\section{Extracting images}

Cpdf can extract the raster images to a given location. JPEG, JPEG2000 and JBIG2 images are extracted directly. Other images are written as PNGs, processed with either ImageMagick's ``magick'' command, or NetPBM's ``pnmtopng'' program, whichever is installed.

  \begin{framed}
  \noindent\small\verb@cpdf -extract-images in.pdf [<range>] [-im <path>] [-p2p <path]@
  \noindent\small\verb@     [-dedup | -dedup-perpage] -o <path>@
  \end{framed}

\noindent The \texttt{-im} or \texttt{-p2p} option is used to give the path to the external tool, one of which must be installed. The output specifer, e.g \verb!-o output/%%%! gives the number format for numbering the images. Output files are named serially from 0, and include the page number too. For example, output files might be called \texttt{output/000-p1.jpg}, \texttt{output/001-p1.png}, \texttt{output/002-p3.jpg} etc. Here is an example invocation:

  \begin{framed}
  \noindent\small\verb@cpdf -extract-images in.pdf -im magick -o output/%%%@
  \end{framed}

\noindent The \texttt{output} directory must already exist. The \texttt{-dedup} option deduplicates images entirely; the \texttt{-dedup-perpage} option only per page.

  \section{Detecting Low-resolution Images}\label{imageres}
  To list all images in the given range of pages which fall below a given resolution (in dots-per-inch), use the \verb!-image-resolution! function:
  \begin{framed}
  \noindent\small\verb@cpdf -image-resolution 300 in.pdf [<range>]@
  \end{framed}

  \begin{framed}
{\small\begin{verbatim}2, /Im5, 531, 684, 149.935297, 150.138267
2, /Im6, 184, 164, 149.999988, 150.458710
2, /Im7, 171, 156, 149.999996, 150.579145
2, /Im9, 65, 91, 149.999986, 151.071856
2, /Im10, 94, 60, 149.999990, 152.284285
2, /Im15, 184, 139, 149.960011, 150.672060
4, /Im29, 53, 48, 149.970749, 151.616446\end{verbatim}}
  \end{framed}
  \noindent The format is \textit{page number, image name, x pixels, y pixels, x resolution, y resolution}. The resolutions refer to the image's effective resolution at point of use (taking account of scaling, rotation etc).

\section{Removing an Image}

To remove a particular image, find its name using \texttt{-image-resolution} with a sufficiently high resolution (so as to list all images), and then apply the \texttt{-draft} and \texttt{-draft-remove-only} operations from Section \ref{draft}. 

\begin{cpdflib}
\clearpage
\section*{C Interface}
\begin{small}\tt
\lstinputlisting{splits/c14}
\end{small}
\end{cpdflib}

\begin{pycpdflib}
\clearpage
\section*{Python Interface}
\begin{small}\tt
\lstinputlisting{pysplits/c14}
\end{small}
\end{pycpdflib}
\pagestyle{fancy}

\begin{dotnetcpdflib}
\clearpage
\section*{.NET Interface}
\begin{small}\tt
\lstinputlisting{dotnetsplits/c14}
\end{small}
\end{dotnetcpdflib}

\begin{jcpdflib}
\clearpage
\section*{Java Interface}
\begin{small}\tt
\lstinputlisting{javasplits/c14}
\end{small}
\end{jcpdflib}

\begin{jscpdflib}
\clearpage
\section*{JavaScript Interface}
\begin{small}\tt
\lstinputlisting{javascriptsplits/c14}
\end{small}
\end{jscpdflib}

\chapter{Fonts}\pagestyle{fancy}\label{chap:14}
 {\small \begin{framed}

  \small\noindent\verb!cpdf -list-fonts in.pdf!

  \vspace{1.5mm}
  \noindent\verb!cpdf -print-font-table <font name> -print-font-table-page <n> in.pdf!
  
  \vspace{1.5mm}
  \noindent\verb!cpdf -copy-font fromfile.pdf -copy-font-page <int>!\\
  \noindent\verb!     -copy-font-name <name> in.pdf [<range>] -o out.pdf!

  \vspace{1.5mm}
  \noindent\verb!cpdf -remove-fonts in.pdf -o out.pdf!

  \vspace{1.5mm}
  \noindent\verb!cpdf -missing-fonts in.pdf!

  \vspace{1.5mm}
  \noindent\verb!cpdf -embed-missing-fonts -gs <path to gs> in.pdf -o out.pdf!
  \end{framed}}

\section{Listing Fonts}

\index{font!listing}
\label{listingfonts}
  The \texttt{-list-fonts} operation prints the fonts in the document,
one-per-line to standard output. For example:
\begin{framed}\small\begin{verbatim}1 /F245 /Type0 /Cleargothic-Bold /Identity-H
1 /F247 /Type0 /ClearGothicSerialLight /Identity-H
1 /F248 /Type1 /Times-Roman /WinAnsiEncoding
1 /F250 /Type0 /Cleargothic-RegularItalic /Identity-H
2 /F13 /Type0 /Cleargothic-Bold /Identity-H
2 /F16 /Type0 /Arial-ItalicMT /Identity-H
2 /F21 /Type0 /ArialMT /Identity-H
2 /F58 /Type1 /Times-Roman /WinAnsiEncoding
2 /F59 /Type0 /ClearGothicSerialLight /Identity-H
2 /F61 /Type0 /Cleargothic-BoldItalic /Identity-H
2 /F68 /Type0 /Cleargothic-RegularItalic /Identity-H
3 /F47 /Type0 /Cleargothic-Bold /Identity-H
3 /F49 /Type0 /ClearGothicSerialLight /Identity-H
3 /F50 /Type1 /Times-Roman /WinAnsiEncoding
3 /F52 /Type0 /Cleargothic-BoldItalic /Identity-H
3 /F54 /Type0 /TimesNewRomanPS-BoldItalicMT /Identity-H
3 /F57 /Type0 /Cleargothic-RegularItalic /Identity-H
4 /F449 /Type0 /Cleargothic-Bold /Identity-H
4 /F451 /Type0 /ClearGothicSerialLight /Identity-H
4 /F452 /Type1 /Times-Roman /WinAnsiEncoding
\end{verbatim}
\end{framed}

\noindent The first column gives the page number, the second the internal unique font
name, the third the type of font (Type1, TrueType etc), the fourth the PDF font
name, the fifth the PDF font encoding.

\section{Listing characters in a font}
\index{font!print table for}
We can use \texttt{cpdf} to find out which characters are available in a given font, and to print the map between character codes, unicode codepoints, and Adobe glyph names. This is presently a best-effort service, and does not cover all font/encoding types.

We find the name of the font by using \texttt{-list-fonts}:

{\small\begin{verbatim}
$ ./cpdf -list-fonts cpdfmanual.pdf 1
1 /F46 /Type1 /XYPLPB+NimbusSanL-Bold 
1 /F49 /Type1 /MCBERL+URWPalladioL-Roma 
\end{verbatim}}

We may then print the table, giving either the font's name (e.g \texttt{/F46}) or basename (e.g \texttt{/XYPLPB+NimbusSanL-Bold}):

{\small\begin{verbatim}
$ ./cpdf -print-font-table /XYPLPB+NimbusSanL-Bold
         -print-font-table-page 1 cpdfmanual.pdf
67 = U+0043 (C - LATIN CAPITAL LETTER C) = /C
68 = U+0044 (D - LATIN CAPITAL LETTER D) = /D
70 = U+0046 (F - LATIN CAPITAL LETTER F) = /F
71 = U+0047 (G - LATIN CAPITAL LETTER G) = /G
76 = U+004C (L - LATIN CAPITAL LETTER L) = /L
80 = U+0050 (P - LATIN CAPITAL LETTER P) = /P
84 = U+0054 (T - LATIN CAPITAL LETTER T) = /T
97 = U+0061 (a - LATIN SMALL LETTER A) = /a
99 = U+0063 (c - LATIN SMALL LETTER C) = /c
100 = U+0064 (d - LATIN SMALL LETTER D) = /d
101 = U+0065 (e - LATIN SMALL LETTER E) = /e
104 = U+0068 (h - LATIN SMALL LETTER H) = /h
105 = U+0069 (i - LATIN SMALL LETTER I) = /i
108 = U+006C (l - LATIN SMALL LETTER L) = /l
109 = U+006D (m - LATIN SMALL LETTER M) = /m
110 = U+006E (n - LATIN SMALL LETTER N) = /n
111 = U+006F (o - LATIN SMALL LETTER O) = /o
112 = U+0070 (p - LATIN SMALL LETTER P) = /p
114 = U+0072 (r - LATIN SMALL LETTER R) = /r
115 = U+0073 (s - LATIN SMALL LETTER S) = /s
116 = U+0074 (t - LATIN SMALL LETTER T) = /t
\end{verbatim}}

The first column is the character code, the second the Unicode codepoint, the character itself and its Unicode name, and the third the Adobe glyph name.

\section{Copying Fonts}
\label{copyfont}

In order to use a font other than the standard 14 with \verb!-add-text!, it
must be added to the file. The font source PDF is given, together with the
font's resource name on a given page, and that font is copied to all the pages
in the input file's range, and then written to the output file.

The font is named in the output file with its basefont name, so it can be
easily used with \verb!-add-text!.

For example, if the file \verb!fromfile.pdf! has a font \verb!/GHLIGA+c128! with
the name \verb!/F10! on page 1 (this information can be found with
\verb!-list-fonts!), the following would copy the font to the file
\verb!in.pdf! on all pages, writing the output to \verb!out.pdf!:
  \begin{framed}
  \small\noindent\verb!cpdf -copy-font fromfile.pdf -copy-font-name /F10!\\
  \small\noindent\verb!     -copy-font-page 1 in.pdf -o out.pdf!
  \end{framed}

\noindent Text in this font can then be added by giving \verb!-font /GHLIGA+c128!. Be
aware that due to the vagaries of PDF font handling concerning which characters
are present in the source font, not all characters may be available, or cpdf may not be able to work out the conversion from UTF8 to the font's own encoding. You may add \texttt{-raw} to the command line to avoid any conversion, but the encoding (mapping from input codes to glyphs) may be non-obvious and require knowledge of the PDF format to divine.

\section{Removing Fonts}
\label{removefont}

To remove embedded fonts from a document, use \verb!-remove-fonts!. PDF readers will
substitute local fonts for the missing fonts. The use of this function is only
recommended when file size is the sole consideration.

  \begin{framed}
  \small\noindent\verb!cpdf -remove-fonts in.pdf -o out.pdf!
  \vspace{2.5mm}
  \end{framed}

\section{Missing Fonts}
  The \verb!-missing-fonts! operation lists any unembedded fonts in the document, one per line.
  \begin{framed}
  \small\noindent\verb!cpdf -missing-fonts in.pdf!
  \end{framed}

  \noindent The format is
  \begin{framed}
  \small\noindent\verb!Page number, Name, Subtype, Basefont, Encoding!
  \end{framed}

\noindent The operation \texttt{-embed-missing-fonts} will process the file with \texttt{gs} (which must be installed) to embed missing fonts (where found):

  \begin{framed}
  \small\noindent\verb!cpdf -embed-missing-fonts -gs gs in.pdf -o out.pdf!
  \end{framed}

\label{listmisingfonts}\clearpage\pagestyle{empty}

\begin{cpdflib}
\clearpage
\section*{C Interface}
\begin{small}\tt
\lstinputlisting{splits/c15}
\end{small}
\end{cpdflib}

\begin{pycpdflib}
\clearpage
\section*{Python Interface}
\begin{small}\tt
\lstinputlisting{pysplits/c15}
\end{small}
\end{pycpdflib}

\begin{dotnetcpdflib}
\clearpage
\section*{.NET Interface}
\begin{small}\tt
\lstinputlisting{dotnetsplits/c15}
\end{small}
\end{dotnetcpdflib}

\begin{jcpdflib}
\clearpage
\section*{Java Interface}
\begin{small}\tt
\lstinputlisting{javasplits/c15}
\end{small}
\end{jcpdflib}

\begin{jscpdflib}
\clearpage
\section*{JavaScript Interface}
\begin{small}\tt
\lstinputlisting{javascriptsplits/c15}
\end{small}
\end{jscpdflib}

\chapter{PDF and JSON}\label{chap:15}\pagestyle{fancy}\index{JSON}\index{JSON!output to}\index{JSON!input from}
  {\small\begin{framed}
  \noindent\verb!cpdf in.pdf -output-json -o out.json!\\
  \noindent\verb!     [-output-json-parse-content-streams]!\\
  \noindent\verb!     [-output-json-no-stream-data]!\\
  \noindent\verb!     [-output-json-decompress-streams]!\\
  \noindent\verb!     [-output-json-clean-strings]!

\vspace{1.5mm}

  \noindent\verb!cpdf -j in.json -o out.pdf!
  \end{framed}}

In addition to reading and writing PDF files in the original Adobe format, \texttt{cpdf} can read and write them in its own CPDFJSON format, for somewhat easier extraction of information, modification of PDF files, and so on.

\section{Converting PDF to JSON}

We convert a PDF file to JSON format like this:

  \begin{framed}
  \small\noindent\verb!cpdf -output-json in.pdf -o out.json!
  \end{framed}

The resultant JSON file is an array of arrays containing an object number followed by an
object, one for each object in the file and two special ones:

\begin{itemize}
\item Object -1: CPDF's own data with the PDF version number, CPDF JSON format
number, and flags used when writing (which may be required when reading):

\begin{itemize}
  \item \texttt{/CPDFJSONformatversion} (CPDFJSON integer (see below), currently 2)
  \item \texttt{/CPDFJSONcontentparsed} (boolean, true if content streams have been parsed)
  \item \texttt{/CPDFJSONstreamdataincluded} (boolean, true if stream data included. Cannot
  round-trip if false).
  \item \texttt{/CPDFJSONmajorpdfversion} (CPDFJSON integer)
  \item \texttt{/CPDFJSONminorpdfversion} (CPDFJSON integer)
\end{itemize}

\item Object 0: The PDF's trailer dictionary

\item Objects 1..n: The PDF's objects.
\end{itemize}

\noindent Objects are formatted thus:

\begin{itemize}
  \item PDF arrays, dictionaries, booleans, and strings are the same as in JSON.
  \item Integers are written as \texttt{\{"I":\ 0\}}
  \item Floats are written as \texttt{\{"F":\ 0.0\}}
  \item Names are written as \texttt{\{"N":\ "/Pages"\}}
  \item Indirect references are integers
  \item Streams are \texttt{\{"S":\ [dict, data]\}}
  \item Strings are converted to JSON string format in a way which, when reversed, results in the original string.
\end{itemize}

\noindent Here is an example of the output for a small PDF:

{\small\begin{verbatim}
[
  [
    -1,
    { "/CPDFJSONformatversion": { "I": 2 },
      "/CPDFJSONcontentparsed": false,
      "/CPDFJSONstreamdataincluded": true,
      "/CPDFJSONmajorpdfversion": { "I": 1 },
      "/CPDFJSONminorpdfversion": { "I": 1 } }
  ],
  [
    0,
    { "/Size": { "I": 4 }, "/Root": 4,
      "/ID" : [ <elided>, <elided>] } ],
  [
    1, { "/Type": { "N": "/Pages" }, "/Kids": [ 3 ], "/Count": { "I": 1 } }
  ],
  [
    2,
    {"S": [{ "/Length": { "I": 49 } },
     "1 0 0 1 50 770 cm BT/F0 36 Tf(Hello, World!)Tj ET"] }
  ],
  [
    3, { "/Type": { "N": "/Page" }, "/Parent": 1,
    "/Resources": {
      "/Font": {
        "/F0": {
          "/Type": { "N": "/Font" },
          "/Subtype": { "N": "/Type1" },
          "/BaseFont": { "N": "/Times-Italic" }
        }
      }
    },
    "/MediaBox":
      [{ "I": 0 }, { "I": 0 },
       { "F": 595.2755905510001 }, { "F": 841.88976378 }],
    "/Rotate": { "I": 0 },
    "/Contents": [ 2 ] } ],
[
  4, { "/Type": { "N": "/Catalog" }, "/Pages": 1 } ]
]\end{verbatim}}

\noindent The option \texttt{-output-json-parse-content-streams} will also convert content streams to JSON, so our example content stream will be expanded:


{\small\begin{verbatim}
2, {
"S": [
  {}, [
  [
  { "F": 1.0 }, { "F": 0.0 }, { "F": 0.0 }, { "F": 1.0 }, { "F": 50.0 }, {
  "F": 770.0 }, "cm" ], [ "BT" ], [ "/F0", { "F": 36.0 }, "Tf" ], [
  "Hello, World!", "Tj" ], [ "ET" ] ]
] } ], [
\end{verbatim}}

\noindent The option \texttt{-output-json-no-stream-data} simply elides the stream data instead, 
leading to much smaller JSON files. 

The option \texttt{-output-json-decompress-streams} keeps the streams intact, and decompresses them.

The option \texttt{-output-json-clean-strings} converts any UTF16BE strings with no high bytes to PDFDocEncoding prior to output, so that editing them is easier. 

\section{Converting JSON to PDF}

We can load a JSON PDF file with the \texttt{-j} option in place of a PDF file anywhere in a normal \texttt{cpdf} command. A range may be applied, just like any other file. 

  \begin{framed}
  \small\noindent\verb!cpdf -j in.json -o out.pdf!
  \end{framed}

It is not required that \texttt{/Length} entries in CPDFJSON stream dictionaries be correctly updated when the JSON file is edited: \texttt{cpdf} will fix them when loading.

\begin{cpdflib}
\clearpage
\section*{C Interface}
\begin{small}\tt
\lstinputlisting{splits/c16}
\end{small}
\end{cpdflib}

\begin{pycpdflib}
\clearpage
\section*{Python Interface}
\begin{small}\tt
\lstinputlisting{pysplits/c16}
\end{small}
\end{pycpdflib}

\begin{dotnetcpdflib}
\clearpage
\section*{.NET Interface}
\begin{small}\tt
\lstinputlisting{dotnetsplits/c16}
\end{small}
\end{dotnetcpdflib}

\begin{jcpdflib}
\clearpage
\section*{Java Interface}
\begin{small}\tt
\lstinputlisting{javasplits/c16}
\end{small}
\end{jcpdflib}

\begin{jscpdflib}
\clearpage
\section*{JavaScript Interface}
\begin{small}\tt
\lstinputlisting{javascriptsplits/c16}
\end{small}
\end{jscpdflib}

\clearpage\pagestyle{empty}
\chapter{Optional Content Groups}\label{chap:16}\pagestyle{fancy}\index{optional content group}

  {\small\begin{framed}
  \noindent\verb!cpdf -ocg-list in.pdf!

  \vspace{1.5mm}
  \noindent\verb!cpdf -ocg-rename -ocg-rename-from <a> -ocg-rename-to <b> in.pdf -o out.pdf!

  \vspace{1.5mm}
  \noindent\verb!cpdf -ocg-order-all in.pdf -o out.pdf!

  \vspace{1.5mm}
  \noindent\verb!cpdf -ocg-coalesce-on-name in.pdf -o out.pdf!

  \end{framed}}


In a PDF file, optional content groups are used to group graphical elements together, so they may appear or not, depending on the preference of the user. They are similar in some ways to layers in graphics illustration programs.

  {\small\begin{framed}
  \noindent\verb!cpdf -ocg-list in.pdf!
  \end{framed}}

\noindent List the optional content groups in the PDF, one per line, to standard output. UTF8.
  
  {\small\begin{framed}
  \noindent\verb!cpdf -ocg-rename -ocg-rename-from <a> -ocg-rename-to <b> in.pdf -o out.pdf!
  \end{framed}}

\noindent Rename an optional content group.



  {\small\begin{framed}
  \noindent\verb!cpdf -ocg-coalesce-on-name in.pdf -o out.pdf!
  \end{framed}}


\noindent Coalesce optional content groups. For example, if we merge or stamp two files both with an OCG called "Layer 1", we will have two different optional content groups. Running \texttt{-ocg-coalesce-on-name} will merge the two into a single optional content group.

  {\small\begin{framed}
  \noindent\verb!cpdf -ocg-order-all in.pdf -o out.pdf!
  \end{framed}}

\noindent Ensure that every optional content group appears in the order list.

\begin{cpdflib}
\clearpage
\section*{C Interface}
\begin{small}\tt
\lstinputlisting{splits/c17}
\end{small}
\end{cpdflib}

\begin{pycpdflib}
\clearpage
\section*{Python Interface}
\begin{small}\tt
\lstinputlisting{pysplits/c17}
\end{small}
\end{pycpdflib}

\begin{dotnetcpdflib}
\clearpage
\section*{.NET Interface}
\begin{small}\tt
\lstinputlisting{dotnetsplits/c17}
\end{small}
\end{dotnetcpdflib}

\begin{jcpdflib}
\clearpage
\section*{Java Interface}
\begin{small}\tt
\lstinputlisting{javasplits/c17}
\end{small}
\end{jcpdflib}

\begin{jscpdflib}
\clearpage
\section*{JavaScript Interface}
\begin{small}\tt
\lstinputlisting{javascriptsplits/c17}
\end{small}
\end{jscpdflib}

\clearpage\pagestyle{empty}
\chapter{Creating New PDFs}\label{chap:17}\pagestyle{fancy}\index{Create}

  {\small\begin{framed}
  \noindent\verb!cpdf -create-pdf [-create-pdf-pages <n>]!\\
  \noindent\verb!     [-create-pdf-papersize <paper size>] -o out.pdf!

  \vspace{1.5mm}
  \noindent\verb!cpdf -typeset <text file> [-create-pdf-papersize <size>]!\\
  \noindent\verb!     [-font <font>] [-font-size <size>] -o out.pdf!

  \end{framed}}

\section{A new blank PDF}
\index{create new PDF}

We can build a new PDF file, given a number of pages and a paper size. The default is one page, A4 portrait.

\begin{framed}
 \small\verb?cpdf -create-pdf -create-pdf-pages 20?\\
 \noindent\small\verb?        -create-pdf-papersize usletterportrait -o out.pdf?
\end{framed}

\noindent The standard paper sizes are listed in Section \ref{papersizes}, or you may specify the width and height directly, as described in the same chapter.

\section{Convert a text file to PDF}
\index{text!convert to PDF}
A basic text to PDF convertor is included in \texttt{cpdf}. It takes a UTF8 text file (ASCII is a subset of UTF8) and typesets it ragged-right, splitting on whitespace. Both Windows and Unix line endings are allowed. 

\begin{framed}
 \small\verb?cpdf -typeset file.txt -create-pdf-papersize a3portrait?\\
 \noindent\small\verb?        -font Courier -font-size 10 -o out.pdf?
\end{framed}

\noindent The standard paper sizes are listed in Section \ref{papersizes}, or you may specify the width and height directly, as described in the same chapter. The standard fonts are listed in chapter~\ref{chap:8}. The default font is TimesRoman and the default size is 12.

\begin{cpdflib}
\clearpage
\section*{C Interface}
\begin{small}\tt
\lstinputlisting{splits/c18}
\end{small}
\end{cpdflib}

\begin{pycpdflib}
\clearpage
\section*{Python Interface}
\begin{small}\tt
\lstinputlisting{pysplits/c18}
\end{small}
\end{pycpdflib}

\begin{dotnetcpdflib}
\clearpage
\section*{.NET Interface}
\begin{small}\tt
\lstinputlisting{dotnetsplits/c18}
\end{small}
\end{dotnetcpdflib}

\begin{jcpdflib}
\clearpage
\section*{Java Interface}
\begin{small}\tt
\lstinputlisting{javasplits/c18}
\end{small}
\end{jcpdflib}

\begin{jscpdflib}
\clearpage
\section*{JavaScript Interface}
\begin{small}\tt
\lstinputlisting{javascriptsplits/c18}
\end{small}
\end{jscpdflib}

\clearpage\pagestyle{empty}
%We wanted to call this "Chapter M", but the following commands messed up the PDF bookmarks, so this chapter will simply have to float for now, until we can return to this problem.
%\setcounter{chapter}{12}
%\renewcommand{\thechapter}{\Alph{chapter}}%
\chapter{Miscellaneous}\label{chap:misc}\pagestyle{fancy}
  {\small\begin{framed}
  \noindent\verb!cpdf -draft [-boxes] [-draft-remove-only <n>] in.pdf [<range>] -o out.pdf!

  \vspace{1.5mm}
  \noindent\verb!cpdf -remove-all-text in.pdf [<range>] -o out.pdf!

  \vspace{1.5mm}
  \noindent\verb!cpdf -blacktext in.pdf [<range>] -o out.pdf!

  \vspace{1.5mm}
  \noindent\verb!cpdf -blacklines in.pdf [<range>] -o out.pdf!

  \vspace{1.5mm}
  \noindent\verb!cpdf -blackfills in.pdf [<range>] -o out.pdf!

  \vspace{1.5mm}
  \noindent\verb!cpdf -thinlines <minimum thickness> in.pdf [<range>] -o out.pdf!

  \vspace{1.5mm}
  \noindent\verb!cpdf -clean in.pdf -o out.pdf!

  \vspace{1.5mm}
  \noindent\verb!cpdf -set-version <version number> in.pdf -o out.pdf!

  \vspace{1.5mm}
  \noindent\verb!cpdf -copy-id-from source.pdf in.pdf -o out.pdf!

  \vspace{1.5mm}
  \noindent\verb!cpdf -remove-id in.pdf -o out.pdf!

  \vspace{1.5mm}
  \noindent\verb!cpdf -list-spot-colors in.pdf!

  \vspace{1.5mm}
  \noindent\verb!cpdf -print-dict-entry <key> in.pdf!

  \vspace{1.5mm}
  \noindent\verb!cpdf -remove-dict-entry <key> [-dict-entry-search <term>]!\\
  \noindent\verb!      in.pdf -o out.pdf!

  \vspace{1.5mm}
  \noindent\verb!cpdf -replace-dict-entry <key> -replace-dict-entry-value <value>!\\
  \noindent\verb!     [-dict-entry-search <term>] in.pdf -o out.pdf!

  \vspace{1.5mm}
  \noindent\verb!cpdf -remove-clipping [<range>] in.pdf -o out.pdf!
  \end{framed}}
  \section{Draft Documents}
\index{draft}
\label{draft}
    The \texttt{-draft} operation removes bitmap (photographic) images from a
file, so that it can be printed with less ink. Optionally, the
\texttt{-boxes} option can be added, filling the spaces left blank with a
crossed box denoting where the image was. This is not guaranteed to be fully
visible in all cases (the bitmap may be have been partially covered by vector
objects or clipped in the original). For example:
  \begin{framed}
    \small\verb!cpdf -draft -boxes in.pdf -o out.pdf!
  \end{framed}

\noindent To remove a single image only, specify \texttt{-draft-remove-only}, giving the name of the image obtained by a call to \texttt{-image-resolution} as described in Section \ref{imageres} and giving the appropriate page. For example:

  \begin{framed}
    \small\verb!cpdf -draft -boxes -draft-remove-only "/Im1" in.pdf 7 -o out.pdf!
  \end{framed}

\noindent To remove text instead of images, use the \texttt{-remove-all-text} operation:

  \begin{framed}
    \small\verb!cpdf -remove-all-text in.pdf -o out.pdf!
  \end{framed}

  \section{Blackening Text, Lines and Fills}
\index{blacken!text}
  Sometimes PDF output from an application (for instance, a web browser) has
text in colors which would not print well on a grayscale printer. The
\texttt{-blacktext} operation blackens all text on the given pages so it will be readable
when printed.

  This will not work on text which has been converted to outlines, nor on text
which is part of a form.
  \begin{framed}
    \small\verb!cpdf -blacktext in.pdf -o out.pdf!
  \end{framed}

\index{blacken!lines}

\noindent The \texttt{-blacklines} operation blackens all lines on the given pages.

  \begin{framed}
    \small\verb!cpdf -blacklines in.pdf -o out.pdf!
  \end{framed}

\index{blacken!fills}

\noindent The \texttt{-blackfills} operation blackens all fills on the given pages.

  \begin{framed}
    \small\verb!cpdf -blackfills in.pdf -o out.pdf!
  \end{framed}

\noindent Contrary to their names, all these operations can use another color, if specified with \texttt{-color}.

  \section{Hairline Removal}
\index{hairline removal}
  Quite often, applications will use very thin lines, or even the value of 0,
which in PDF means "The thinnest possible line on the output device". This
might be fine for on-screen work, but when printed on a high resolution device,
such as by a commercial printer, they may be too faint, or disappear
altogether. The \texttt{-thinlines} operation prevents this by changing all lines
thinner than \texttt{<minimal~thickness>} to the given thickness. For example:
  \begin{framed}
  \small\noindent\verb!cpdf -thinlines 0.2mm in.pdf [<range>] -o out.pdf!

  \vspace{2.5mm}
  \noindent Thicken all lines less than 0.2mm to that value.
  \end{framed} 

  \section{Garbage Collection}
\index{garbage collection}
  Sometimes incremental updates to a file by an application, or bad
applications can leave data in a PDF file which is no longer used. This
function removes that unneeded data.

  \begin{framed}
  \small\noindent\verb!cpdf -clean in.pdf -o out.pdf!
  \end{framed}
 
  \section{Change PDF Version Number}
\index{version number}
   \label{setversion}
   To change the pdf version number, use the \texttt{-set-version} operation,
giving the part of the version number after the decimal point. For example:
  \begin{framed}
  \small\noindent\verb!cpdf -set-version 4 in.pdf -o out.pdf!

  \vspace{2.5mm}
  \noindent Change file to PDF 1.4.
  \end{framed} 
  \noindent This does not alter any of the actual data in the file ---
just the supposed version number. For PDF versions starting with 2 add ten to the number. For example, for PDF version 2.0, use \texttt{-set-version 10}.

  \section{Copy ID}
\index{file ID!copy}
  The \texttt{-copy-id-from} operation copies the ID from the given file to the
input, writing to the output.
  \begin{framed}
  \small\noindent\verb!cpdf -copy-id-from source.pdf in.pdf -o out.pdf!

  \vspace{2.5mm}
  \noindent Copy the id from \texttt{source.pdf} to the contents of \texttt{in.pdf}, writing to \texttt{out.pdf}.
  \end{framed}
  \noindent If there is no ID in the source file, the existing ID is retained. You cannot use \texttt{-recrypt} with \texttt{-copy-id-from}.

\section{Remove ID}
\index{file ID!remove}
  The \texttt{-remove-id} operation removes the ID from a document.
  \begin{framed}
  \small\noindent\verb!cpdf -remove-id in.pdf -o out.pdf!

  \vspace{2.5mm}
  \noindent Remove the ID from \texttt{in.pdf}, writing to \texttt{out.pdf}.
  \end{framed}

\noindent You cannot use \texttt{-recrypt} with \texttt{-remove-id}.

\section{List Spot Colours}
\index{spot colour}
This operation lists the name of any ``separation'' color space in the given PDF file.

  \begin{framed}
  \small\noindent\verb!cpdf -list-spot-colors in.pdf!

  \vspace{2.5mm}
  \noindent List the spot colors, one per line in \texttt{in.pdf}, writing to \texttt{stdout}.
  \end{framed}

\section{PDF Dictionary Entries}
\label{removedictentry}
This is for editing data within the PDF's internal representation. Use with caution.

\index{dictionary!print entry}
\index{dictionary!remove entry}
\index{dictionary!replace entry}

To print a dictionary entry:

  \begin{framed}
  \small\noindent\verb!cpdf -print-dict-entry /URI in.pdf -o out.pdf!

  \vspace{2.5mm}
  \noindent Print all URLs in annotation hyperlinks \texttt{in.pdf}. 
  \end{framed}

To remove a dictionary entry:

  \begin{framed}
  \small\noindent\verb!cpdf -remove-dict-entry /One in.pdf -o out.pdf!

  \vspace{2.5mm}
  \noindent Remove the entry for \texttt{/One} in every dictionary \texttt{in.pdf}, writing to \texttt{out.pdf}. 

  \vspace{2.5mm}

  \small\noindent\verb!cpdf -remove-dict-entry /One -dict-entry-search "1" in.pdf -o out.pdf!

  \vspace{2.5mm}
  \noindent Replace the entry for \texttt{/One} in every dictionary \texttt{in.pdf} if the key's value is the given value, writing to \texttt{out.pdf}. 
  \end{framed}

To replace a dictionary entry, give the replacement value in CPDFJSON format:

  \begin{framed}
  \small\noindent\verb!cpdf -replace-dict-entry /One -replace-dict-entry-value "\{I : 2\}"!\\
  \small\noindent\verb!     in.pdf -o out.pdf!

  \vspace{2.5mm}
  \noindent Remove the entry for \texttt{/One} in every dictionary \texttt{in.pdf}, writing to \texttt{out.pdf}. 

  \vspace{2.5mm}

  \small\noindent\verb!cpdf -replace-dict-entry /One -dict-entry-search "\{I : 1\}"!\\
  \small\noindent\verb!     -replace-dict-entry-value "\{I : 2\}" in.pdf -o out.pdf!

  \vspace{2.5mm}
  \noindent Remove the entry for \texttt{/One} in every dictionary \texttt{in.pdf} if the key's value is the given value, writing to \texttt{out.pdf}. 
  \end{framed}

\section{Removing Clipping}


The \texttt{-remove-clipping} operation removes any clipping paths on given pages from the file.

  \begin{framed}
  \small\noindent\verb!cpdf -remove-clipping in.pdf -o out.pdf!

  \vspace{2.5mm}
  \noindent Remove clipping paths in \texttt{in.pdf}, writing to \texttt{out.pdf}. 
  \end{framed}

\begin{cpdflib}
\clearpage
\section*{C Interface}
\begin{small}\tt
\lstinputlisting{splits/c19}
\end{small}
\end{cpdflib}

\begin{pycpdflib}
\clearpage
\section*{Python Interface}
\begin{small}\tt
\lstinputlisting{pysplits/c19}
\end{small}
\end{pycpdflib}

\begin{dotnetcpdflib}
\clearpage
\section*{.NET Interface}
\begin{small}\tt
\lstinputlisting{dotnetsplits/c19}
\end{small}
\end{dotnetcpdflib}

\begin{jcpdflib}
\clearpage
\section*{Java Interface}
\begin{small}\tt
\lstinputlisting{javasplits/c19}
\end{small}
\end{jcpdflib}

\begin{jscpdflib}
\clearpage
\section*{JavaScript Interface}
\begin{small}\tt
\lstinputlisting{javascriptsplits/c19}
\end{small}
\end{jscpdflib}

\appendix
\chapter{Dates}\pagestyle{empty}
\label{dates}
\index{date!defined}

\section{PDF Date Format}
Dates in PDF are specified according to the following format:

\begin{framed}
\verb!D:YYYYMMDDHHmmSSOHH'mm'!\\\\where:

\begin{itemize}
  \item \texttt{YYYY} is the year;
  \item \texttt{MM} is the month;
  \item \texttt{DD} is the day (01-31);
  \item \texttt{HH} is the hour (00-23);
  \item \texttt{mm} is the minute (00-59);
  \item \texttt{SS} is the second (00-59);
  \item \texttt{O} is the relationship of local time to Universal Time (UT), denoted by '+', '-' or 'Z';
  \item \texttt{HH} is the absolute value of the offset from UT in hours (00-23);
  \item \texttt{mm} is the absolute value of the offset from UT in minutes (00-59).
\end{itemize}
\end{framed}

\noindent A contiguous prefix of the parts above can be used instead, for lower
accuracy dates. For example:

\begin{framed}
   \small\noindent\verb!D:2014! (2014)
   
   \vspace{1.5mm}
   \noindent\verb!D:20140103! (3rd January 2014)

   \vspace{1.5mm}
   \noindent\verb!D:201401031854-08'00'! (3rd January 2014, 6:54PM, US Pacific Standard Time)
   
\end{framed}

\section{XMP Metadata Date Format}
\label{xmpdate}


These are the possible data formats for \texttt{-set-metadata-date}:

\begin{verbatim}
YYYY
YYYY-MM
YYYY-MM-DD
YYYY-MM-DDThh:mmTZD
YYYY-MM-DDThh:mm:ssTZD
\end{verbatim}

\noindent where:

\medskip
\begin{tabular}{ll}
\texttt{YYYY} & year \\
\texttt{MM} & month (01 = Jan)\\
\texttt{DD} & day of month (01 to 31)\\
\texttt{hh} & hour (00 to 23)\\
\texttt{mm} & minute (00 to 59)\\
\texttt{ss} & second (00 to 59)\\
\texttt{TZD} & time zone designator (\texttt{Z} or \texttt{+hh:mm} or \texttt{-hh::mm})
\end{tabular}
\pagestyle{fancy}


\section{CPDF Change Log}
{\footnotesize\begin{alltt}
\input{Changes}
\end{alltt}}

\section{CamlPDF Change Log}
(CamlPDF is the library CPDF is based upon)

{\footnotesize\begin{alltt}
\input{../camlpdf/Changes}
\end{alltt}}


\backmatter
\pagestyle{fancy}

\printindex
\end{document}

